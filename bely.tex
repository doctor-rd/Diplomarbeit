%\documentclass[a4paper]{article}
\documentclass{article}
%\documentclass[a4paper,oneside]{book}
%\hyphenation{ab-ge-schlos-sen-en}
\newtheorem{thm}{Theorem}[subsection]
\newtheorem{df}[thm]{Definition}
\newtheorem{lem}[thm]{Lemma}
\newtheorem{satz}[thm]{Satz}
\newtheorem{bem}[thm]{Bemerkung}
\newtheorem{fol}[thm]{Folgerung}
\def\disj{\mkern+13mu\cdot\mkern-13mu\bigcup}
\def\C{\mathbbm{C}}
\def\Q{\mathbbm{Q}}
\def\Palg{\mathbbm{P}^1}
\def\P1{\Palg_\C}
\def\Pc{\Palg_C}
\def\alg{\Q^{\mathop{\rm alg}}}
\def\nat{\mathbbm{N}}
\def\Nat{\mathbbm{N^+}}
\def\bew{Beweis:\\}
\def\iv{In\-duktions\-voraussetzung}
\def\ex{existiert}
\def\zz{Zu zeigen: }
\def\impl{\:\Longrightarrow\:}
%\def\bem{\textbf{Bemerkung}\\}
\def\gb{gebrochen-linear}
\def\lu{linear unabh\"angig}
\def\Tf{Transformation}
\def\tf{transformier}
%
\def\var{Variet\"at}
\def\KE{K\"orpererweiterung}
\def\ghom{Gruppen\homo}
\def\khom{K\"orper\homo}
\def\rhom{Ring\homo}
\def\homo{homomorphismus}
\def\algabg{algebraisch abgeschlossen}
%\def\deg{\textrm{grad }}
\def\sieg{\begin{flushright}$\Box$\end{flushright}}
\def\dreier{\{0,1,\infty\}}
%
\def\mod{\mathop{\rm mod}\nolimits}
\def\Aut{\mathop{\rm Aut}\nolimits}
\def\Der{\mathop{\rm Der}\nolimits}
\def\supp{\mathop{\rm supp}\nolimits}
\def\id{\mathop{\rm id}\nolimits}
\def\Im{\mathop{\rm Im}\nolimits}
\def\Spec{\mathop{\rm Spec}\nolimits}
\def\hom{\mathop{\rm Hom}\nolimits}
\def\proj{\mathop{\rm Proj}\nolimits}
\def\max{\mathop{\rm max}\limits}
\def\stab{\mathop{\rm Stab}\limits}
%
\newenvironment{bsp}{\vspace{3mm}\ \newline\textbf{Beispiele}\begin{enumerate}}{\end{enumerate}}
\newenvironment{minilemma}{\smallskip\\\em}{\smallskip\\}
\def\O{{\cal O}}
\def\p{{\textfrak p}}
\def\m{{\textfrak m}}
\usepackage{bbm,amssymb}
\usepackage[latin1]{inputenc}
%\usepackage[T1]{fontenc} <- spackt ab
\usepackage{t1enc}
\usepackage[all,cmtip]{xy}
\usepackage{oldgerm} % f�r Primideale
\usepackage{german}
\begin{document}
\begin{titlepage}
\begin{center}
{\Large Diplomarbeit}\\
\vspace{5mm}
{\huge\textsc{Satz von Belyi}}\\
\vspace{5cm}
im Studiengang Mathematik\\der Universit�t Bremen\\vorgelegt im\\
\vspace{6mm}
{\Large Januar 2008}\\
\vspace{6mm}
von\\
\vspace{6mm}
{\Large Ralf Donau}
\end{center}
\vspace{7cm}
Gutachter:\\
Prof. Dr. Jens Gamst\\
Prof. Dr. Eberhard Oeljeklaus
\end{titlepage}
\tableofcontents
\section{Einleitung}
Eine projektive Kurve $X$ �ber einem \algabg en K�rper $C$ ist �ber einem Unterk�rper $K\subset C$ definiert, falls $X$ eine affine �berdeckung hat, so da� jeder affine Teil isomorph zu einer L�sungsmenge von algebraischen Gleichungen �ber $K$ ist. Der Satz von Belyi sagt aus, da� eine komplexe projektive glatte Kurve $X$ genau dann �ber einem Zahlk�rper definiert ist, falls ein nicht konstanter Morphismus $X\longrightarrow\P1$ \ex, der h�chstens drei kritische Werte besitzt. In dieser Arbeit werde ich einen Beweis dieses Satzes durchf�hren, dabei richte ich mich nach dem Artikel ``Belyi's Theorem Revisited'' von B. K�ck.

Der Beweis des Satzes von Belyi hat zwei Richtungen. Einerseits m�ssen wir zu einer Kurve, die bereits �ber einem Zahlk�rper definiert ist, einen Morphismus nach $\P1$ angeben, der h�chstens drei kritische Werte hat. Das wird in Kapitel \ref{abschnitt_belyverf} durchgef�hrt. Andererseits, wenn ein Morphismus $X\longrightarrow\P1$ gegeben ist, der h�chstens drei kritische Werte besitzt, ist zu zeigen, da� $X$ �ber einer endlichen Erweiterung von $\Q$ definiert ist, also �ber einem Zahlk�rper. Diese Richtung wird h�ufig als ``obvious part'' bezeichnet, was zutreffen mag, wenn einem die Resultate von Weil bekannt sind. Den Beweis dieser Richtung zerlegt B.\ K�ck in zwei Teile, dazu benutzt er den Modulk�rper von $t$, der in Kapitel \ref{abschnitt_modulk} definiert wird, als Zwischenschritt. Im ersten Schritt zeigt er, da� der Modulk�rper von $t$ eine endliche Erweiterung von $\Q$ ist. Im zweiten Schritt wird gezeigt, da� die Kurve �ber einer endlichen Erweiterung des Modulk�rpers definiert ist. Zuvor hatte J. Wolfart in seinem Beweis den absoluten Modulk�rper der Kurve $X$ verwendet, allerdings ist der Beweis von B. K�ck mit dem Modulk�rper von $t$ einfacher.

Im Folgenden gebe ich eine kurze �bersicht �ber den Beweis, der in dieser Arbeit behandelt wird. Die einzelnen Schritte werden in der Sprache der Schemata durchgef"uhrt, dazu verwende ich folgende Formulierung des Satzes von Belyi:
\begin{thm}
Sei $X$ eine vollst�ndige Kurve �ber $\C$. $X$ ist genau dann �ber einem Zahlk�rper definiert, wenn ein endlicher Morphismus $t:X\longrightarrow\P1$ \ex, der h�chstens $3$ kritische Werte besitzt.
\end{thm}
Die einzelnen Schritte des Beweises werden in den Kapiteln \ref{abschnitt_belyverf}, \ref{abschnitt_modulk} und \ref{abschnitt_defk} ausgef�hrt.

Zuerst nehmen wir an, da� ein Morphismus $t:X\longrightarrow\P1$ mit h�chstens drei kritischen Werten \ex. Nachdem wir $t$ eventuell mit einer \gb en \Tf\ komponiert haben, k�nnen wir annehmen, da� die kritischen Werte von $t$ in $\dreier$ liegen. Der Modulk�rper $M(X,t)$ von $t$ ist nach Lemma \ref{koeck32} eine endliche Erweiterung von $\Q$. $X$ und $t$ sind nach Theorem \ref{koeck22} �ber einer endlichen Erweiterung von $M(X,t)$ definiert, damit ist $X$ insgesamt �ber einer endlichen Erweiterung von $\Q$, also einem Zahlk�rper definiert.

Falls $X$ �ber einem Zahlk�rper definiert ist, liefert uns das Belyi Verfahren aus Kapitel \ref{abschnitt_belyverf} den gew�nschten Morphismus.

{\large Konventionen:}
\begin{itemize}
\item Alle betrachteten K�rper haben Charakteristik 0
\item Ringe sind kommutativ mit Einselement
\item Alle Kurven sind glatt
\end{itemize}
\newpage
\section{Vorbereitungen I}
Im folgenden ist $C$ ein algebraisch abgeschlossener K�rper und alle auftretenden K�rper haben Charakteristik 0.
\subsection{\var en}
%Q Hartshorne S.82
\begin{df}
Ein Schema $X$ hei�t integral, falls f�r jedes offene $U\subset X$ der Ring $\O_{X,U}$ ein Integrit�tsring ist.
\end{df}
\begin{df}
Eine \var\ �ber einem K�rper $K$ (kurz $K$-\var) ist ein integrales Schema $X$ mit einem Strukturmorphismus $$X\stackrel{\varphi}{\longrightarrow}\Spec K$$ welcher separiert und von endlichem Typ ist.
\end{df}
``von endlichem Typ'' bedeutet in diesem Fall, da� $X$ eine offene �berdeckung durch $(\Spec A_i)_{i=1\dots n}$ hat wobei jedes $A_i$ eine endlich erzeugte $K$-Algebra ist.
Schreibweise: Wenn wir von der \var\ $X\stackrel{\varphi}{\longrightarrow}\Spec K$ sprechen, bezeichnen wir diese kurz mit $X$.
\begin{bem}
Sei $\O_X$ die Strukturgarbe von $X$, dann gilt f�r jede offene affine Teilmenge $U\subset X$:
\begin{enumerate}
\item $\O_X(U)$ ist eine endlich erzeugte $K$-Algebra.
\item F�r $V\subset U$ offen ist die Restriktion $\O_X(U)\longrightarrow\O_X(V)$ ein $K$-Algebren\-homo\-morphismus.
\end{enumerate}
\end{bem}
\begin{bsp}
%Q Hart S.82, S.84, S.96
\item Affine \var en: Sei $A$ eine endlich erzeugte $K$-Algebra, die ein Integrit�tsring ist, d.h. $A=K[X_1\cdots X_n]/I$, wobei $I$ ein Primideal in $K[X_1\cdots X_n]$ ist. Dann ist das affine Schema $\Spec A$ zusammen mit dem durch die Inklusion $K\hookrightarrow A$ induzierten Morphismus $\Spec A\longrightarrow\Spec K$ eine $K$-\var.

\item $\Pc$ als projektives Schema zu einem algebraisch abgeschlossenen K�rper $C$.
\end{bsp}
\begin{df}
Seien $X$ und $Y$ \var en �ber einem K�rper $K$ mit den Strukturmorphismen $\varphi$ und $\psi$. Ein Morphismus von Schemata $f:X\longrightarrow Y$ hei�t \textbf{Morphismus von $K$-\var en}, falls folgendes Diagramm kommutiert:
$$
\xymatrix{
X \ar[rr]^f \ar[rd]_\varphi &&Y\ar[ld]^\psi\\
&\Spec K}
$$
\end{df}
%Q Hart S.16
%Q Forster S.40
\begin{df}
Sei $X$ ein integrales Schema, $\O_X$ die zugeh�rige Strukturgarbe. Eine rationale Funktion auf $X$ wird durch ein Paar $(U,f)$ repr�sentiert, wobei $U\subset X$ offen und nicht leer ist und $f\in O_{X,U}$ gilt. Wir identifizieren zwei durch $(U,f)$ und $(V,g)$ repr�sentierte rationale Funktionen miteinander, falls $f_{|U\cap V}=g_{|U\cap V}$ gilt. Die Menge der rationalen Funktionen auf $X$ bezeichnen wir mit $K(X)$.
\end{df}
Man sieht leicht, da� die auf den Paaren $(U,f)$ definierte Relation eine �quivalenzrelation ist.
\begin{bem}
$K(X)$ ist ein K�rper.
\end{bem}
Wenn $X$ eine $K$-\var\ ist, so ist $K(X)$ eine $K$-Algebra. Die Einbettung\\ $K\hookrightarrow K(X)$ h�ngt vom Strukturmorphismus $\varphi:X\longrightarrow\Spec K$ ab. $K(X)$ selbst h�ngt lediglich vom zugrunde liegenden Schema $X$ ab.
%Q hart s.24
\begin{df}
Seien $X$ und $Y$ irreduzibe Schemata. Eine rationale Abbildung von $X$ nach $Y$ wird durch ein Paar $(U,f)$ repr�sentiert, wobei $U$ ein offenes Unterschema von $X$ ist und $U\stackrel{f}{\longrightarrow}Y$ ein Morphimus von Schemata. Wir identifizieren zwei durch $(U,f)$ und $(V,g)$ repr�sentierte rationale Abbildungen miteinander, falls $f_{|U\cap V}=g_{|U\cap V}$. D.h. das Diagramm
$$
\xymatrix{
&U\cap V\ar@{_{(}->}[rd]\ar@{^{(}->}[ld]\\
U\ar[rd]^f&&V\ar[ld]_g\\
&Y
}
$$
kommutiert.
\end{df}
\begin{df}
Seien $X$ und $Y$ $K$-\var en und $X\stackrel{f}{\longrightarrow}Y$ eine rationale Abbildung. $f$ hei�t �ber $K$ definiert, falls $f$ durch einen Morphismus von $K$-\var en repr�sentiert wird.
\end{df}
Jede rationale Abbildung $X\longrightarrow Y$, die durch einen Morphismus repr�sentiert wird, dessen stetiges Bild dicht ist, induziert einen \khom\ $K(Y)\longrightarrow K(X)$ zwischen den Funktionenk�rpern. Jeden Morphismus zwischen Schemata kann man als rationale Abbildung auffassen.\bigskip\\
In Kapitel \ref{abschnitt_modulk} wird der Modulk�rper eines endlichen Morphismus $X\longrightarrow\Pc$ definiert, dazu ben�tigen wir die folgende Definition:
\begin{df}
\label{modulk_dep}
Sei $\sigma\in\Aut(K)$ ein K�rperautomorphimus, und sei $X$ eine $K$-\var\ mit dem Strukturmorphismus $\varphi:X\longrightarrow\Spec K$. Dann erhalten wir durch Kompositum mit $\Spec\sigma:\Spec K\longrightarrow\Spec K$ eine neue $K$-\var\ mit dem Strukturmorphismus $X\stackrel{\varphi}{\longrightarrow}\Spec K\stackrel{\Spec\sigma}{\longrightarrow}\Spec K$. Diese bezeichnen wir mit $X^\sigma$.
\end{df}
Die zugrundeliegenden Schemata von $X$ und $X^\sigma$ sind identisch. Die $K$-\var en $X$ und $X^\sigma$ unterscheiden sich lediglich in den Strukturmorphismen nach $\Spec K$.

Sei $t:X\longrightarrow Y$ ein Morphismus zwischen $K$-\var en, und sei $\sigma\in\Aut(K)$. Dann ist auch $t:X^\sigma\longrightarrow Y^\sigma$ ein Morphismus zwischen $K$-\var en, wie man leicht sieht. Wir bezeichnen diesen Morphismus dann mit $t^\sigma$.
\medskip\\\textbf{Beispiel:}\\
Sei $X$ eine affine Variet�t, d.h. $X=\Spec A$ mit $A=K[X_1\cdots X_n]/I$. Dann ist die \var\ $X^\sigma$ isomorph zur \var\ $\Spec B$ mit $B=K[X_1\cdots X_n]/\sigma^{-1}(I)$.
\begin{df}
Sei $L$ ein K�rper, und sei $X$ eine \var\ �ber $L$. Sei\\ $K\subset L$ ein Unterk�rper, dann hei�t $X$ �ber $K$ definiert, falls eine $K$-\var\ $X_K$ \ex, so da� $$X=X_K\times_{\Spec K}\Spec L$$ gilt.
\end{df}
%Q AlgGeo 1 hinter NoethNorm, ist ein Satz
%Q Ernst Kunz S.131 Koro 3.9
\begin{bem}
\label{defkldim}
$X$ und $X_K$ haben die gleiche Dimension.
\end{bem}
\begin{df}
Seien $X$ und $Y$ \var en �ber $L$, die bereits �ber $K$ definiert sind. Sei $f:X\longrightarrow Y$ ein Morphismus von $L$-\var en. Dann hei�t $f$ �ber $K$ definiert, falls ein Morphismus $f_K:X_K\longrightarrow Y_K$ von $K$-\var en \ex, so da� folgendes Diagramm kommutiert:
%\[\xymatrix{
%&X_K\ar[ld]\ar[dd]_{f_K}&&X\ar[ll]\ar[dd]^f\ar[rd]\\
%\Spec K&&&&\Spec L\\
%&Y_K\ar[lu]&&Y\ar[ll]\ar[ru]
%}\]
\[\xymatrix{
&X_K\ar[d]_{f_K}&&X\ar[ll]\ar[d]^f\\
&Y_K&&Y\ar[ll]
}\]
\end{df}
\begin{df}
Eine \textbf{(glatte) Kurve} �ber $C$ ist eine $C$-\var\ $X$, deren zugrundeliegendes Schema eindimensional ist, wobei f�r jeden abgeschlossenen Punkt $x\in X$ der lokale Ring $\O_{X,x}$ ein diskreter Bewertungsring ist.
\end{df}
Wir betrachten ausschlie�lich glatte Kurven, daher wird der Zusatz ``glatt'' weggelassen.\medskip\\
Sei $X$ eine Kurve �ber $C$ und $K(X)$ deren K�rper der rationalen Funktionen von $X$, dann gilt $deg_{tr}(K(X):C)=1$. Die \KE\ $K(X):C(T)$ ist endlich, aus dem Satz vom primitiven Element folgt dann, da� $K(X)$ �ber $C$ von h�chstens zwei Elementen erzeugt wird, d.h. $K(X)=C(a,b)$ mit $a,b\in K(X)$. Umgekehrt \ex\ zu jeder endlich erzeugten \KE\ $\widetilde C$ vom Transzendenzgrad $1$ �ber $C$ eine vollst�ndige Kurve $Y$, so da� $K(Y)=\widetilde C$ gilt.
\begin{df}
Sei $X$ ein (glatte) Kurve, dann hei�t $X$ \textbf{vollst�ndig}, falls zu jedem diskreten Bewertungsring $R$, der zwischen $K$ und $K(X)$ liegt, ein \mbox{$x\in X$} \ex, so da� $R\cong \O_{X,x}$ gilt.
\end{df}
%Q Ha S.43
\begin{satz}
Seien $X$ und $Y$ vollst�ndige Kurven �ber $K$, und sei $S\subset X$ eine endliche Teilmenge von abgeschlossenen Punkten. Sei $\varphi:X\setminus S\longrightarrow Y$ ein Morphismus von $K$-\var en, dann \ex\ eindeutig eine Fortsetzung \\$\widetilde\varphi:X\longrightarrow Y$ von $\varphi$.
\end{satz}
\begin{fol}
Jede rationale Abbildung zwischen zwei vollst�ndigen Kurven $X$ und $Y$, die �ber $C$ definiert ist, wird von einem Morphismus $X\longrightarrow Y$ repr�sentiert. Ein \khom\ $K(Y)\longrightarrow K(X)$ induziert einen Morphismus $X\longrightarrow Y$.
\end{fol}
\subsection{Endliche Morphismen}
%Q Ha S.91 Ex 3.4
\begin{df}
Seien $X$ und $Y$ $C$-\var en, und sei $f:X\longrightarrow Y$ ein Morphismus. Dann ist $f$ ein \textbf{endlicher Morphismus}, falls f�r jede offene affine Teilmenge $V\cong\Spec A\subset Y$ das Urbild $U:=f^{-1}(V)$ affin ist, also $U\cong\Spec B$ gilt, und $B$ verm�ge dem induzierten Morphismus $A\longrightarrow B$ eine modulendliche $A$-Algebra ist.
\end{df}
Sei $f:X\longrightarrow Y$ ein endlicher Morphismus zwischen Kurven �ber $C$, dann ist das Bild jeder offenen Teilmenge dicht in $Y$. Damit induziert $f$ einen \khom\ $f^*:K(Y)\longrightarrow K(X)$, dieser ist injektiv. Insbesondere sind damit die induzierten Ringhomomorphismen zwischen den affinen Koordinatenringen und auch die lokalen Ringhomomorphismen injektiv.
\begin{satz}
\label{welchemorphendl}
Seien $X$ und $Y$ Kurven, und sei $X$ vollst�ndig. Sei $f:X\longrightarrow Y$ ein nicht-konstanter Morphismus. Dann gilt
\begin{enumerate}
\item $f$ ist ein endlicher Morphismus
\item $f(X)=Y$
\item $Y$ ist vollst�ndig.
\end{enumerate}
\end{satz}
Beweis: Proposition 6.8 auf S.137 in \cite{Hartshorne}\bigskip\\
Seien $X$ und $Y$ vollst�ndige Kurven, und sei $f$ ein endlicher Morphismus. Dann ist der Grad der \KE\ $K(X):K(Y)$ endlich. Wir definieren den \textbf{Grad} von $f$ als den Grad der \KE\ $K(X):K(Y)$. Wir wissen, da� sich Grade von \KE en multiplizieren, d.h. seien $L:M$ und $M:K$ endliche \KE en, dann gilt $[L:K]=[K:M]\cdot[M:K]$. Per Definition multiplizieren sich die Grade von endlichen Morphismen ebenfalls.
\begin{df}
Seien $X$ und $Y$ Kurven �ber $C$, und sei $t:X\longrightarrow Y$ ein endlicher Morphismus. $\Aut(t)$ bezeichne die Gruppe derjenigen Isomorphismen $f:X\longrightarrow X$, f�r die $t\circ f=t$ gilt.
\[
\xymatrix{
X\ar[rr]^f \ar[rd]_t &&X\ar[ld]^t\\
&Y}
\]
\end{df}
\begin{lem}
\label{algcovgalois}
Seien $X$ und $Y$ Kurven �ber $C$, und sei $t:X\longrightarrow Y$ ein endlicher Morphismus. Sei $P\in X$, so da� $t$ unverzweigt in $P$ ist. Sei $f:X\longrightarrow X$ aus $\Aut(t)$ mit $f(P)=P$, dann ist $f$ die Identit�t.
\end{lem}
\[\xymatrix{
X \ar[rr]^f \ar[rd]_t &&X\ar[ld]^t\\
&Y
}\]
\bew
$t$ bzw. $f$ induziert einen lokalen $C$-Algebren\homo\\ $\varphi_t:\O_{Y,f(P)}\longrightarrow\O_{X,P}$ bzw. $\varphi_f:\O_{X,P}\longrightarrow\O_{X,P}$ im Punkt $P$. Sei $r\in\O_{Y,f(P)}$ ein lokaler Parameter, dann ist $s:=\varphi_t(r)$ ebenfalls ein lokaler Parameter, da $t$ unverzweigt in $P$ ist.
\[\xymatrix{
\O_{X,P}&&\O_{X,P}\ar[ll]_{\varphi_f}\\
&\O_{Y,f(P)}\ar[ul]^{\varphi_t}\ar[ur]_{\varphi_t}
}\]
Es gilt
\[
\varphi_f(s)=\varphi_f(\varphi_t(r))=\varphi_t(r)=s
\]
Damit ist $\varphi_f$ die Identit�t, da $\varphi_f$ durch das Bild des lokalen Parameters $s$ eindeutig bestimmt ist. $f$ induziert einen \khom\\ $f^*:K(X)\longrightarrow K(X)$ dieser ist die Identit�t, da $K(X)$ der Quotientenk�rper von $\O_{X,P}$ ist. Damit ist auch $f$ die Identit�t.\sieg
\begin{df}
Seien $p_1:Y_1\longrightarrow X$ und $p_2:Y_1\longrightarrow X$ endliche Morphismen zwischen Kurven �ber $\C$. $(Y_1,p_1)$ und $(Y_2,p_2)$ hei�en isomorph als endliche  Morphismen nach $X$, falls ein Isomorphismus $f:Y_1\longrightarrow Y_2$ \ex, so da� $p_2\circ f=p_1$ gilt.
\[\xymatrix{
Y_1\ar[rd]_{p_1}\ar[rr]^f&&Y_2\ar[ld]^{p_2}\\
&X
}\]
\end{df}
\begin{bem}
Die soeben definierte Relation zwischen endlichen Morphismen ist eine �quivalenzrelation.
\end{bem}
Die Begr�ndung ist die gleiche wie die von Bemerkung \ref{eqrelisossdfjko}.\sieg

\def\deri{Derivation}
\def\crit{\mathop{\rm Crit}\nolimits}
\def\ann{\mathop{\rm Ann}\nolimits}
\section{Differentiale}
In diesem Abschnitt werden die kritischen Punkte von Morphismen zwischen Kurven definiert. Das Ziel ist dann, die Menge der kritischen Punkte eines endlichen Morphismus zu kennzeichnen. Wir werden sehen, da� diese mit dem Tr�ger einer Garbe �bereinstimmt und sogar endlich ist.

Ich erinnere daran, da� bei uns alle K�rper Charakteristik $0$ haben. Im Folgenden ist $C$ ein \algabg er K�rper und $K$ ein nicht notwendig \algabg er K�rper.
\subsection{Tangential- und Cotangentialraum}
\begin{df}
Sei $A$ ein Ring, $B$ eine $A$-Algebra und $M$ ein $B$-Modul. Eine Abbildung $d:B\longrightarrow M$ hei�t $A$-\textbf{\deri}, falls folgendes gilt:
\begin{enumerate}
\item $d$ ist additiv
\item $d$ gen�gt der Produktregel
\item $d(a)=0$ f�r alle $a\in A$
\end{enumerate}
Die Menge aller $A$-\deri en $d:B\longrightarrow M$ bezeichnen wir mit $\Der_A(B,M)$.
\end{df}
\begin{df}
Sei $A$ ein Ring, $B$ eine $A$-Algebra, dann ist der \textbf{Modul der relativen Differentialformen} von $B$ �ber $A$ ein $B$-Modul $\Omega_{B/A}$ mit einer $A$-\deri\ $d_{B/A}:B\longrightarrow\Omega_{B/A}$, welche folgende universelle Eigenschaft hat:
\begin{minilemma}
Zu jeder $A$-\deri\ $v$ in einen $B$-Modul $M$ \ex\ eindeutig eine $B$-lineare Abbildung $w:\Omega_{B/A}\longrightarrow M$, so da� $w\circ d_{B/A}=v$ gilt.
\end{minilemma}
\[
\xymatrix{
B\ar[rrd]_v\ar[rr]^{d_{B/A}}&&\Omega_{B/A}\ar@{.>}[d]^w\\
&&M
}
\]
\end{df}
Eine solche \deri\ $d_{B/A}$ kann man zu jeder $A$-Algebra $B$ konstruieren.
\bigskip\\
%Q Ha S.40
Sei $A$ eine lokale $K$-Algebra mit maximalem Ideal $\m$, und sei $v:A\longrightarrow M$ eine $K$-\deri, dann gilt nach der Produktregel $v(\m^2)\subset\m M$, denn
\[
a,b\in\m\Longrightarrow v(ab)=av(b)+bv(a)\in\m M
\]
Wenn nun $M=A/\m$ gilt, ist der K�rper $M$ verm�ge der Quotientenabbildung ein $A$-Modul, und es gilt $\m M=0$. Das hei�t, die im Folgenden h�ufig auftretenden \deri en $v:A\longrightarrow A/\m$ verschwinden auf $\m^2$.
\begin{lem}
\label{disk_supertrick}
Sei $A$ eine lokale $K$-Algebra mit maximalem Ideal $\m$. Sei \mbox{$K\cong A/\m$} als $K$-Algebren, dann hat jedes $a\in A$ eine eindeutige Darstellung $a=a_1+a_2$ mit $a_1\in K$ und $a_2\in\m$.
\end{lem}
\bew
Setze $a_1:=(a\mod\m)$ (man denke an den $K$-Algebrenisomorphismus) und \mbox{$a_2:=a-a_1$}. Dann gilt $a_1\in K$, und es gilt $a_2\in\m$, denn
\[(a_2\mod\m)=(a\mod\m)-(a_1\mod\m)=(a\mod\m)-(a\mod\m)=0\]
Aus $K\cap\m=\{0\}$ folgt die Eindeutigkeit dieser Darstellung.\sieg
%Q Alg Geo 2 Affine Schemata Handout S.10
\begin{bem}
\label{tauschenquotlokal}
Sei $A$ ein kommutativer Ring, $S\subset A$ ein multiplikatives System, $A_S$ die Lokalisierung nach $S$ und $j:A\longrightarrow A_S$ die kanonische ``Einbettung''. Dann gibt $\p\longmapsto j(\p)A_S$ eine Bijektion zwischen den Primidealen $\p\subset A$ mit $\p\cap S=\emptyset$ und den Primidealen von $A_S$. Sei $q:A\longrightarrow A/\p$ die Quotientenabbildung, dann gilt $A_S/j(\p)A_S\cong(A/\p)[q(S)]$.
\end{bem}
\begin{df}
Sei $X$ eine $C$-\var, $x\in X$ ein abgeschlossener Punkt und $\O_{X,x}$ der Halm in $x$. Sei $\m_x\subset\O_{X,x}$ das maximale Ideal, dann hei�t
\begin{enumerate}
\item $\Der_C(\O_{X,x},C)$ der \textbf{Tangentialraum} von $X$ im Punkt $x$.
\item $\m_x/\m_x^2$ der \textbf{Cotangentialraum} von $X$ im Punkt $x$.
\end{enumerate}
\end{df}
\begin{bem}
Es gilt $C\cong\O_{X,x}/\m_x$ und damit ist $C$ �ber die Quotientenabbildung $q:\O_{X,x}\longrightarrow C$ ein $\O_{X,x}$-Modul.
\end{bem}
\bew
$\O_{X,x}$ ist die Lokalisierung einer endlich erzeugten $C$-Algebra, d.h. $\O_{X,x}\cong A_{\m}$, wobei $A$ eine endlich erzeugte $C$-Algebra sei und $\m\subset A$ ein maximales Ideal. Aus dem Hilbertschen Nullstellensatz folgt $C\cong A/\m$. Nach Bemerkung \ref{tauschenquotlokal} gilt
\[A_{\m}/\m A_{\m}\cong A/\m\]
Insgesamt erhalten wir
\[\O_{X,x}/\m_x\cong A_{\m}/\m A_{\m}\cong A/\m\]
\sieg
\begin{satz}
\label{dualcotandemisttandem}
Der Tangential- und Cotangentialraum sind $C$-Vektorr�ume und es gilt
\[\hom_C(\m_x/\m_x^2,C)=\Der_C(\O_{X,x},C)\]
\end{satz}
\bew
Sei $v\in\Der_C(\O_{X,x},C)$, dann gilt $v(\m_x^2)=0$. Wir erhalten eine Linearform $v':\m_x/\m_x^2\longrightarrow C$, indem wir $v$ repr�sentantenweise operieren lassen. Man sieht leicht, da� diese Zuordnung $C$-linear ist.

Diese Zuordnung ist auch injektiv, denn sei $v'=0$, dann gilt $v_{|\m_x}=0$. Zu zeigen ist nun $v=0$. Sei $a\in\O_{X,x}$, dann k�nnen wir $a$ nach Lemma \ref{disk_supertrick} als Summe aus $a_1\in C$ und $a_2\in\m_x$ schreiben. Also gilt
\[v(a)=v(a_1)+v(a_2)=0+0=0\]

Jetzt m�ssen wir noch sehen, da� diese Zuordnung surjektiv ist. Sei \\\mbox{$w:\m_x/\m_x^2\longrightarrow C$} eine Linearform. Wir definieren eine Abbildung $v:\O_{X,x}\longrightarrow C$ wie folgt: F�r $a=a_1+a_2$ mit $a_1\in C$ und $a_2\in\m_x$ setze $v(a):=w(\overline{a_2})$. Dann gilt trivialerweise $v(c)=0$ f�r $c\in C$. Seien nun $a=a_1+a_2$ und $b=b_1+b_2$ aus $\O_{X,x}$. Dann gilt
\begin{eqnarray*}
v(ab)&=&v((a_1+a_2)(b_1+b_2))\\
&=&v(a_1 b_2)+v(a_2 b_1)\\
&=&a_1 v(b_2)+b_1 v(a_2)\\
&=&a v(b)+b v(a)
\end{eqnarray*}
Also ist $v$ eine $C$-\deri\ auf $\O_{X,x}$. Es gilt $v'=w$, damit ist die Surjektivit�t gezeigt.\sieg
%Q L�t S. 249
\begin{fol}
Sei $X$ eine Kurve (Erinnerung: wir betrachten ausschlie�lich glatte Kurven), $x\in X$ ein abgeschlossener Punkt, dann sind der Tangentialraum und der Cotangentialraum eindimensionale $C$-Vektorr�ume.
\end{fol}
\bew
$\O_{X,x}$ ist ein diskreter Bewertungsring, also ist $\m/\m^2$ ein eindimensionaler $C$-Vektor\-raum. Nach Satz \ref{dualcotandemisttandem} ist der Tangentialraum der Dualraum vom Cotan\-gential\-raum, also ist auch der Tangentialraum eindimensional.\sieg
Sei $\varphi:A\longrightarrow B$ ein $K$-Algebrenhomomorphismus, und sei $v:B\longrightarrow M$ eine $K$-\deri\ in einen $B$-Modul $M$. Verm�ge $\varphi$ ist $M$ dann ein $A$-Modul, und $v\circ\varphi:A\longrightarrow M$ ist eine $K$-\deri\ auf $A$, denn seien $a,b\in A$, dann gilt \[v(\varphi(ab))=v(\varphi(a)\varphi(b))=\varphi(b)v(\varphi(a))+\varphi(a)v(\varphi(b))=bv(\varphi(a))+av(\varphi(b))\]
und wegen der $K$-Linearit�t von $\varphi$ gilt $v\circ\varphi(k)=0$ f�r $k\in K$. Additivit�t gilt trivialerweise und damit ist gezeigt, da� $v\circ\varphi$ eine $K$-\deri\ ist.

Seien $A$ und $B$ nun zus�tzlich lokale $K$-Algebren, $\varphi$ ein lokaler $K$-Algebren\-ho\-momorphismus und seien $\m_A$ und $\m_B$ die entsprechenden maximalen Ideale. Es gelte $A/\m_A\cong K\cong B/\m_B$. $K$ ist �ber die Quotientenabbildung \mbox{$q_B:B\longrightarrow B/\m_B$} ein $B$-Modul und ist verm�ge $\varphi$ ein $A$-Modul. Andererseits ist auf $K$ eine $A$-Modulstruktur �ber die Quotientenabbildung $q_A:A\longrightarrow A/\m_A$ erkl�rt. In beiden F�llen erhalten wir dieselbe $A$-Modulstruktur. Um das einzusehen, ist $q_A=q_B\circ\varphi$ zu zeigen (Gleichheit bezieht sich auf die kanonische Identifikation �ber die $K$-Algebrenisomorphismen). Sei $a\in A$, dann k�nnen wir $a$ nach Lemma \ref{disk_supertrick} als Summe aus $a_1\in K$ und $a_2\in\m_A$ schreiben. Es gilt $\varphi(\m_A)\subset\m_B$, also $q_B\circ\varphi(a_2)=0=q_A(a_2)$. $\varphi$ ist ein $K$-Algebrenhomomorphismus, also gilt auch $q_B\circ\varphi(a_1)=q_A(a_1)$. Daraus folgern wir, da� f�r $v\in\Der(B,K)$ die zur�ckgezogene \deri\ $v\circ\varphi$ in $\Der(A,K)$ liegt. Wir k�nnen nun eine Abbildung
\begin{eqnarray*}
\Der(B,K)&\longrightarrow&\Der(A,K)\\
v&\longmapsto&v\circ\varphi
\end{eqnarray*}
definieren, diese Abbildung nennen wir \textbf{Tangentialabbildung}.
\bigskip\\
Seien $X$ und $Y$ $C$-\var en und sei $f:X\longrightarrow Y$ ein Morphismus von $C$-\var en. Sei $x\in X$ ein abgeschlossener Punkt, dann induziert $f$ �ber die Strukturgarben einen lokalen $C$-Algebrenhomomorphismus
$$\varphi:\O_{Y,f(x)}\longrightarrow\O_{X,x}$$
Sei $\m_x$ das maximale Ideal von $\O_{X,x}$ und $\m_{f(x)}$ das maximale Ideal von $\O_{Y,f(x)}$. Es gilt $\varphi(\m_{f(x)})\subset\m_x$, daraus folgt $\varphi(\m_{f(x)}^2)\subset\m_x^2$, also wird in nat�rlicher Weise eine $C$-lineare Abbildung
\[\widetilde\varphi:\m_{f(x)}/\m_{f(x)}^2\longrightarrow\m_x/\m_x^2\]
durch $\varphi$ induziert. $\widetilde\varphi$ hei�t dann die \textbf{Cotangentialabbildung} von $f$ in $x$.
\begin{bem}
\label{tandemabbdef}
Die $C$-lineare Abbildung zwischen den Tangentialr�umen
\begin{eqnarray*}
\Der_C(\O_{X,x},C)&\longrightarrow&\Der_C(\O_{Y,f(x)},C)\\
v&\longmapsto&v\circ\varphi
\end{eqnarray*}
ist die duale Abbildung zu $\widetilde\varphi$.
\end{bem}
Diese Abbildung hei�t \textbf{Tangentialabbildung} von $f$ in $x$.\\
\bew
Sei $v\in\Der_C(\O_{X,x},C)$, dann erhalten wir eine Linearform $v':\m_x/\m_x^2\longrightarrow C$ (siehe Beweis von Satz \ref{dualcotandemisttandem}). \zz $v'\circ\widetilde\varphi=(v\circ\varphi)'$. Sei $\overline\mu\in\m_{f(x)}/\m_{f(x)}^2$ mit $\mu\in\m_{f(x)}$, dann gilt $v\circ\varphi(\mu)=v'\circ\widetilde\varphi(\overline\mu)$.\sieg
Der Cotangentialraum l��t sich auch anders beschreiben und zwar mit dem Modul der relativen Differentialformen. Dazu definieren wir die $\O_{X,x}$-lineare Abbildung
\begin{eqnarray*}
\delta:\m_x&\longrightarrow&\Omega_{\O_{X,x}/C}\\
m&\longmapsto&d_{\O_{X,x}/C}(m)
\end{eqnarray*}
wobei $\Omega_{\O_{X,x}/C}$ den Modul der relativen Differentialformen von $\O_{X,x}$ �ber $C$ bezeichne und $d_{\O_{X,x}/C}$ die zugeh�rige \deri.
Es gilt $\delta(\m_x^2)\subset\m_x\Omega_{\O_{X,x}/C}$, also k�nnen wir eine Abbildung $\widetilde\delta$ definieren:
%Q Ha S.174 Prop 8.7
\begin{satz}
Die $C$-lineare Abbildung
\begin{eqnarray*}
\widetilde\delta:\m_x/\m_x^2&\longrightarrow&\Omega_{\O_{X,x}/C}/\m_x\Omega_{\O_{{X,x}/C}}\\
\overline m&\longmapsto&\overline{d_{\O_{X,x}/C}(m)}
\end{eqnarray*}
ist ein Isomorphismus.
\end{satz}
\bew
Wir definieren eine $C$-\deri\ $v:\O_{X,x}\longrightarrow\m_x/\m_x^2$. Sei $a\in\O_{X,x}$, dann k�nnen wir $a$ nach Lemma \ref{disk_supertrick} als Summe aus $a_1\in C$ und $a_2\in\m_x$ schreiben. Setze $v(a):=a_2\mod\m_x^2$, dann ist $v$ eine $C$-\deri, denn per Definition gilt Additivit�t und $v(c)=0$ f�r $c\in C$. Die Rechnung zum Nachweis der Produktregel ist analog zu der aus dem Beweis von Satz \ref{dualcotandemisttandem}. Zu $v$ \ex\ nun eindeutig ein $w:\Omega_{\O_{{X,x}/C}}\longrightarrow\m_x/\m_x^2$, so da� folgendes Diagramm kommutiert:
\[
\xymatrix{
\O_{X,x}\ar[rrd]_v\ar[rr]^{d_{\O_{X,x}/C}}&&\Omega_{\O_{X,x}/C}\ar@{.>}[d]^w\\
&&\m_x/\m_x^2
}
\]
$\Omega_{\O_{X,x}/C}$ wird von $\{d_{\O_{X,x}/C}(a)\mid a\in\m_x\}$ erzeugt und es gilt\\ $w(d_{\O_{X,x}/C}(a))=v(a)=a\mod\m^2$. Wir schr�nken nun von $\O_{X,x}$ auf $\m_x$ ein und betrachten dann erneut alles modulo $\m_x^2$, damit erhalten wir folgende kommutative Diagramme:
\[
\xymatrix{
\m_x\ar[rrd]_v\ar[rr]^{\delta}&&\Omega_{\O_{{X,x}/C}}\ar[d]^w&\m_x/\m_x^2\ar[rrd]_\id\ar[rr]^{\widetilde\delta}&&\Omega_{\O_{{X,x}/C}}/\m_x\Omega_{\O_{{X,x}/C}}\ar[d]^{\widetilde w}\\
&&\m_x/\m_x^2&&&\m_x/\m_x^2
}
\]
Die Wohldefiniertheit von $\widetilde w$ folgt aus der $\O_{X,x}$-Linearit�t von $w$ und daraus, da� $\m_x\cdot\m_x/\m_x^2=0$ gilt. Nach dem obigen Diagramm ist $\widetilde w$ ein Linksinverses von $\widetilde\delta$. Man sieht leicht, da� $\widetilde\delta\circ\widetilde w$ auf den Erzeugenden die Identit�t ist, damit ist $\widetilde w$ ebenfalls ein Rechtsinverses.\sieg
Verm�ge $\varphi$ ist $\Omega_{\O_{X,x}/C}$ ein $\O_{Y,f(x)}$-Modul und $v:=d_{\O_{X,x}/C}\circ\varphi$ ist eine $C$-\deri\ auf $\O_{Y,f(x)}$. Es \ex\ eindeutig ein $\psi:\Omega_{\O_{Y,f(x)}/C}\longrightarrow\Omega_{\O_{X,x}/C}$, so da� folgendes Diagramm kommutiert.
\[
\xymatrix{
\O_{Y,f(x)}\ar[rrd]_v\ar[rr]^{d_{\O_{Y,f(x)}/C}}&&\Omega_{\O_{{Y,f(x)}/C}}\ar@{.>}[d]^\psi\\
&&\Omega_{\O_{{X,x}/C}}
}
\]
Auf den Erzeugenden $\{d_{\O_{Y,f(x)}/C}(a)\mid a\in\O_{Y,f(x)}\}$ von $\Omega_{\O_{{Y,f(x)}/C}}$ gilt nun \[d_{\O_{Y,f(x)}/C}(a)\stackrel{\psi}{\longmapsto}d_{\O_{X,x}/C}(\varphi(a))\]
Es gilt $\psi(\m_{f(x)}\Omega_{\O_{{Y,f(x)}/C}})\subset\m_x\Omega_{\O_{{X,x}/C}}$, denn sei $mv\in\m_{f(x)}\Omega_{\O_{{Y,f(x)}/C}}$ mit $m\in\m_{f(x)}$ und $v\in\Omega_{\O_{{Y,f(x)}/C}}$, dann gilt $\psi(mv)=\varphi(m)\psi(v)\in\m_x\Omega_{\O_{{X,x}/C}}$, da $\varphi$ ein lokaler \rhom\ ist. Wir k�nnen nun eine Abbildung \[\widetilde\psi:\Omega_{\O_{Y,f(x)}/C}/\m_{f(x)}\Omega_{\O_{{Y,f(x)}/C}}\longrightarrow\Omega_{\O_{X,x}/C}/\m_x\Omega_{\O_{{X,x}/C}}\] in naheliegender Weise definieren. Mit dem so definierten $\widetilde\psi$ kommutiert das folgende Diagramm.
\[
\xymatrix{
\m_{f(x)}/\m_{f(x)}^2\ar[rr]^{\widetilde\varphi}\ar[d]_{\widetilde\delta}&&\m_x/\m_x^2\ar[d]^{\widetilde\delta}\\
\Omega_{\O_{Y,f(x)}/C}/\m_{f(x)}\Omega_{\O_{{Y,f(x)}/C}}\ar[rr]^{\widetilde\psi}&&\Omega_{\O_{X,x}/C}/\m_x\Omega_{\O_{{X,x}/C}}
}
\]
Daher wissen wir nun, da� $\widetilde\varphi$ den gleichen Rang wie $\widetilde\psi$ hat. Seien $X$ und $Y$ Kurven, dann ist $\widetilde\varphi$ genau dann die Nullabbildung, wenn $\widetilde\psi$ die Nullabbildung ist. Das ist wiederum �quivalent dazu, da� die Tangentialabbildung im Punkt $x$ die Nullabbildung ist.
\subsection{Kritische Punkte}
\begin{df}
\label{kritpktdef}
Sei $f:X\longrightarrow Y$ ein Morphismus zwischen Kurven �ber $C$, und sei $x\in X$ ein abgeschlossener Punkt. Dann hei�t $x$ \textbf{kritischer Punkt} von $f$ und $f(x)$ kritischer Wert von $f$, falls eine der folgenden �quivalenten Bedingungen erf�llt ist:
\begin{enumerate}
\item die Tangentialabbildung $\Der_C(\O_{X,x},C)\longrightarrow\Der_C(\O_{Y,f(x)},C)$ aus Satz \ref{tandemabbdef} ist die Nullabbildung
\item Cotangentialabbildung $\widetilde\varphi$ ist die Nullabbildung
\item Cotangentialabbildung auf Differentialformen $\widetilde\psi$ ist die Nullabbildung
\end{enumerate}
\end{df}
Wir bezeichen die Menge aller kritischen Punkte von $f$ mit $\crit(f)$.
\begin{bem}
\label{isokeinkrit}
Ein Isomorphismus zwischen Kurven �ber $C$ hat keine kritischen Punkte.
\end{bem}
\bew
Ein Isomorphismus zwischen lokal geringten R�umen induziert in jedem Punkt einen Isomorphismus zwischen den lokalen Ringen.\sieg
F�r den Beweis von Satz \ref{mats_exakte_seq} ben�tigen wir eine bekannte Tatsache:
%Q Bourbaki Algebra S.227 Thm 1
\begin{satz}
\label{bourhomolog}
Sei $R$ ein Ring, und seien $U,V,W$ $R$-Moduln. Dann ist die Sequenz von $R$-Moduln
\[
U\longrightarrow V\longrightarrow W\longrightarrow 0
\]
genau dann exakt, falls f�r alle $R$-Moduln $M$ die induzierte Sequenz
\[
0\longrightarrow\hom_R(W,M)\longrightarrow\hom_R(V,M)\longrightarrow\hom_R(U,M)
\]
exakt ist.
\end{satz}
Beweis: Theorem 1 auf S.227 in \cite{bour}\bigskip
\begin{lem}
\label{diff_diskret_modul}
Sei $A$ ein lokaler Ring mit maximalem Ideal $\m$, und sei $M$ ein $A$-Modul, der von einem Element erzeugt wird. Ist $x\in M\setminus\m M$, dann erzeugt $x$ den Modul $M$.
\end{lem}
\bew
Sei $y\in M$ ein Erzeugendes von $M$, dann \ex\ $a\in A$ mit $x=ay$. Aus $x\not\in\m M$ folgt $a\not\in\m$, also ist $a$ invertierbar. Nun gilt $y=a^{-1}x$ und damit ist $x$ ein Erzeugendes von $M$.\sieg
%Q Ha S.173 Prop 8.3A
%Q Ma Thm 57 S.186
\begin{satz}
\label{mats_exakte_seq}
Seien $A\stackrel{f}{\longrightarrow}B\stackrel{g}{\longrightarrow}C$ Ringhomomorphismen, dann gibt es in nat�rlicher Weise eine exakte Sequenz von $C$-Moduln.
\[
\Omega_{B/A}\otimes_B C\stackrel{v}{\longrightarrow}\Omega_{C/A}\stackrel{u}{\longrightarrow}\Omega_{C/B}\longrightarrow 0
\]
mit $v(d_{B/A}(b)\otimes 1)=d_{C/A}(b)$ und $u(d_{C/A}(c))=d_{C/B}(c)$.
\end{satz}
\bew
Gem�� Satz \ref{bourhomolog} zeigen wir, da� die exakte Sequenz von $C$-Moduln
\[
0\longrightarrow\hom_C(\Omega_{C/B},M)\stackrel{u^*}{\longrightarrow}\hom_C(\Omega_{C/A},M)\stackrel{v^*}{\longrightarrow}\hom_C(\Omega_{B/A}\otimes_B C,M)
\]
f�r alle $C$-Moduln $M$ exakt ist.

Die $B$-\deri\ $d_{C/B}:C\longrightarrow\Omega_{C/B}$ ist auch eine $A$-\deri, also \ex\ eindeutig ein $C$-lineares $u:\Omega_{C/A}\longrightarrow\Omega_{C/B}$, so da� $u\circ d_{C/A}=d_{C/B}$ gilt, d.h. $u(d_{C/A}(c))=d_{C/B}(c)$ f�r alle $c\in C$.
\[\xymatrix{
C\ar[rr]^{d_{C/A}}\ar[d]_{d_{C/B}}&&\Omega_{C/A}\ar@{.>}[lld]^u\\
\Omega_{C/B}
}\]
Sei nun $M$ ein $C$-Modul. Sei $x\in\hom(\Omega_{C/B},M)$, dann erhalten wir eine $B$-\deri\ $x\circ d_{C/B}\in\Der_B(C,M)$. Aus der Definition des Moduls der relativen Differentialformen folgt, da� diese Zuordnung ein Isomorphismus ist, also gilt \[\Der_B(C,M)\cong\hom(\Omega_{C/B},M)\]
Analog gilt
\[\Der_A(C,M)\cong\hom(\Omega_{C/A},M)\]

Sei $x\in\hom(\Omega_{C/B},M)$, dann gilt $x\circ d_{C/B}=x\circ(u\circ d_{C/A})=u^*(x)\circ d_{C/A}$. Die durch $x$ gegebene $B$-\deri\ stimmt mit der durch $u^*(x)$ gegebenen $A$-\deri\ �berein, also entspricht $u^*$ der Inklusionsabbildung\\ $\Der_B(C,M)\hookrightarrow\Der_A(C,M)$.

Die $A$-\deri\ $d_{C/A}:C\longrightarrow\Omega_{C/A}$ gibt uns eine $A$-\deri\\ $B\stackrel{g}{\longrightarrow}C\stackrel{d_{C/A}}{\longrightarrow}\Omega_{C/A}$. Es \ex\ eindeutig ein $B$-lineares $\widetilde v$, so da� \\$\widetilde v\circ d_{B/A}=d_{C/A}\circ g$ gilt, d.h. $\widetilde v(d_{B/A}(b))=d_{C/A}(b)$ f�r alle $b\in B$.
\[\xymatrix{
B\ar[rr]^{d_{B/A}}\ar[d]_g&&\Omega_{B/A}\ar@{.>}[d]^{\widetilde v}\\
C\ar[rr]^{d_{C/A}}&&\Omega_{C/A}
}\]
Nun machen wir den $B$-Modul $\Omega_{B/A}$ �ber das Tensorprodukt zu einem $C$-Modul. Sei $i:\Omega_{B/A}\longrightarrow\Omega_{B/A}\otimes_B C$ die kanonische ``Inklusion'', dann \ex\ eindeutig ein $C$-lineares $v:\Omega_{B/A}\otimes_B C\longrightarrow\Omega_{C/A}$, so da� $v\circ i=\widetilde v$ gilt.
\[\xymatrix{
\Omega_{B/A}\ar[rr]^i\ar[d]^{\widetilde v}&&\Omega_{B/A}\otimes_B C\ar@{.>}[lld]^v\\
\Omega_{C/A}
}\]
F�r $v$ gilt dann $v(d_{B/A}(b)\otimes 1)=d_{C/A}(b)$ f�r alle $b\in B$.

Sei $x\in\hom_C(\Omega_{B/A}\otimes_B C,M)$, dann gilt $x\circ i\in\hom_B(\Omega_{B/A},M)$. Diese Zuordnung ist ein Isomorphismus. Wir ordnen nun $x\in\hom_C(\Omega_{B/A}\otimes_B C,M)$ eine $A$-\deri\ auf $B$ wie folgt zu: $x\circ i\circ d_{B/A}$.

Sei $x\in\hom_C(\Omega_{C/A},M)$, dann erhalten wir eine $A$-\deri\\ $x\circ d_{C/A}\in\Der_A(C,M)$. Wir schr�nken diese \deri\ auf $B$ ein, dann gilt $x\circ (d_{C/A}\circ g)=x\circ(\widetilde v\circ d_{B/A})=x\circ(v\circ i\circ d_{B/A})=v^*(x)\circ i\circ d_{B/A}$. Wir sehen nun, da� $v^*$ der Einschr�nkung $\Der_A(C,M)\longrightarrow\Der_A(B,M)$ entspricht.

Man sieht sehr leicht, da�
\[
0\longrightarrow\Der_B(C,M)\longrightarrow\Der_A(C,M)\longrightarrow\Der_A(B,M)
\]
exakt ist.\sieg
\begin{satz}
\label{diff_supp_lokal}
Sei $f:X\longrightarrow Y$ ein Morphismus zwischen Kurven, und sei $x\in X$ ein abgeschlossener Punkt. $x$ ist genau dann ein kritischer Punkt von $f$, wenn $\Omega_{\O_{X,x}/\O_{Y,f(x)}}\not=0$ gilt.
\end{satz}
\bew
Wie haben die Ringhomomorphismen $C\longrightarrow\O_{Y,f(x)}\stackrel{\varphi}{\longrightarrow}\O_{X,x}$. Nach Satz \ref{mats_exakte_seq} erhalten wir folgende exakte Sequenz von $\O_{X,x}$-Moduln:
\[\Omega_{\O_{Y,f(x)}/C}\otimes_{\O_{Y,f(x)}}\O_{X,x}\stackrel{v}{\longrightarrow}\Omega_{\O_{X,x}/C}\stackrel{u}{\longrightarrow}\Omega_{\O_{X,x}/\O_{Y,f(x)}}\longrightarrow 0\] wobei $v(d_{\O_{X,x}/C}(f)\otimes a)=d_{\O_{Y,f(x)/C}}\circ\varphi(f)\cdot a$ gilt.

$x$ ist kein kritischer Punkt von $f$ ist �quivalent dazu, da� $\widetilde\psi$ (siehe Definition \ref{kritpktdef}) nicht die Nullabbildung ist. Das wiederum ist �quivalent dazu, da� ein $m$ aus $\Omega_{\O_{X,x}/C}\setminus\m_x\Omega_{\O_{{X,x}/C}}$ unter $\widetilde\psi$ getroffen wird. Nach Lemma \ref{diff_diskret_modul} ist $m$ ein Erzeugendes von $\Omega_{\O_{X,x}/C}$ und damit w�re $v$ surjektiv. Wegen der Exaktheit ist $v$ genau dann surjektiv, wenn $u$ die Nullabbildung ist, und das ist �quivalent zu $\Omega_{\O_{X,x}/\O_{Y,f(x)}}=0$.\sieg
\subsection{Garben von Differentialformen}
%Q Ha S.67 Ex 1.14
\begin{df}
Sei $\cal F$ eine Garbe (von Gruppen, Ringen, Moduln) auf einem topologischen Raum $X$, dann hei�t
\[
\supp{\cal F}:=\{P\in X\mid{\cal F}_P\not=0\}
\]
der \textbf{Tr�ger} von $\cal F$.

Sei $U\subset X$ eine offene Teilmenge und $s\in{\cal F}(U)$, dann hei�t
\[
\supp s:=\{P\in U\mid s_P\not=0\}
\]
der Tr�ger von $s$.
\end{df}
Wir wissen, da� wir zu einem gegebenen Ring $A$ einen topologischen Raum $\Spec A$ mit einer Garbe von Ringen $\O$ konstruieren k�nnen. Sei nun $M$ ein $A$-Modul, dann k�nnen wir eine Garbe $\widetilde M$ von $\O_X$-Moduln, d.h. f�r $U\subset X$ offen ist $\widetilde M(U)$ ein $\O_X(U)$-Modul, auf $\Spec A$ in analoger Weise konstruieren, so da� f�r einen Punkt $\p\in\Spec A$ der Halm von $\widetilde M$ im Punkt $\p$ isomorph zu $M_\p$ ist und f�r eine offene Basismenge $D(f)\subset\Spec A$ der Modul $\widetilde M(D(f))$ isomorph zu $M_f$ ist. Mit den folgenden zwei Lemmata werden wir sehen, da� der Tr�ger einer solchen Garbe abgeschlossen in $\Spec A$ ist.
%Q Ha S.124 5.6a
\begin{lem}
\label{suppabgeschl}
Sei $A$ ein Ring und $M$ ein $A$-Modul. Sei $m\in M$ und $\ann m$ das Ideal $\{a\in A\mid am=0\}$, dann gilt
\[
\supp m=V(\ann m)
\]
\end{lem}
\bew
Sei $\p\in\supp  m$, d.h. das Bild von $m$ in $M_\p$ ist ungleich $0$. Nach der Definition der Lokalisierung von Moduln ist $\frac m1\in M_\p$ genau dann ungleich $\frac 01$, wenn f�r alle $h\in A$ stets $h\not\in\p\Longrightarrow hm\not=0$ gilt. Das ist �quivalent zu $hm=0\Longrightarrow h\in\p$ f�r alle $h\in A$, und das ist eine andere Schreibweise f�r $\ann m\subset\p$. Also gilt $\p\in V(\ann m)$.
\sieg
%Q Ha S.124 5.6b
\begin{lem}
\label{suppabgeschl2}
Sei $A$ ein Ring und $M$ ein endlich erzeugter $A$-Modul. Sei $\ann M$ das Ideal $\{a\in A\mid am=0\emph{ f�r alle }m\in M\}$, dann gilt
\[
\supp\widetilde M=V(\ann M)
\]
\end{lem}
\bew
Seien $m_1,\dots,m_n$ die Erzeugenden von $M$. Es gilt
\begin{enumerate}
\item $\ann M=\bigcap\limits_{i=1}^n\ann(m_i)$
\item $\bigcup\limits_{i=1}^n\supp(m_i)=\supp\widetilde M$
\end{enumerate}
\begin{description}
\item[Zu. 1]\ \\
``$\subset$'' ist klar, es bleibt ``$\supset$'' zu zeigen. Sei $a\in\bigcap\ann(m_i)$ und  $m\in M$, dann hat $m$ eine Darstellung $m=\sum\lambda_i m_i$. Es gilt $a\cdot m=\sum\lambda_i a\cdot m_i=0$. Damit ist $a\in\ann M$ gezeigt.
\item[Zu. 2]\ \\
``$\subset$'' ist klar, es bleibt ``$\supset$'' zu zeigen. Sei $\p\in\supp\widetilde M$, dann \ex\ ein $x\in M$, dessen Bild in $M_\p$ ungleich Null ist. Wir k�nnen $x$ als $x=\sum\lambda_i m_i$ schreiben. F�r ein $i$ ist das Bild von $m_i$ in $M_\p$ ungleich Null, damit liegt $\p$ im Tr�ger von $m_i$. Also gilt $\p\in\bigcup\supp m_i$.
\end{description}
Es gilt
\[V(\ann M)=V(\bigcap\limits_{i=1}^n\ann m_i)=\bigcup\limits_{i=1}^n V(\ann m_i)=\bigcup\limits_{i=1}^n\supp m_i=\supp\widetilde M\]
Die vorletzte Gleichheit folgt aus Lemma \ref{suppabgeschl}.\sieg
\begin{fol}
\label{suppabgeschl3}
Sei $A$ ein Ring und $M$ ein endlich erzeugter $A$-Modul. Dann ist der Tr�ger von $\widetilde M$ abgeschlossen in $\Spec A$.
\end{fol}
%Q Ha S.111
\begin{df}
Sei $X$ ein Schema und $\cal F$ eine Garbe von $\O_X$-Moduln auf $X$, d.h. f�r $U\subset X$ offen ist ${\cal F}(U)$ ein $\O_X(U)$-Modul. Die Garbe $\cal F$ hei�t \textbf{koh�rent}, falls $X$ eine offene affine �berdeckung $(\Spec A_i)_{i\in I}$ hat, so da� zu jedem $i\in I$ ein endlich erzeugter $A_i$-Modul $M_i$ \ex, so da� ${\cal F}(\Spec A_i)\cong\widetilde M_i$ gilt.
\end{df}
Der Tr�ger einer koh�renten Garbe ist nach Folgerung \ref{suppabgeschl3} stets abgeschlossen, da Abgeschlossenheit eine lokale Eigenschaft ist.
%Q Ha S.173 Prop 8.2A
%Q Ma S.186 Ex
\begin{satz}
\label{diff_supp_hilfe}
Seien $A'$ und $B$ $A$-Algebren, und sei $B'=B\otimes_A A'$. Dann gilt
\[\Omega_{B'/A'}\cong\Omega_{B/A}\otimes_B B'\]
Insbesondere gilt f�r eine multiplikativ abgeschlossene Teilmenge $S\subset B$
\[\Omega_{S^{-1}B/A}\cong S^{-1}\Omega_{B/A}\]
\end{satz}
ohne Beweis
\begin{lem}
\label{konstfield}
Seien $A$ und $B$ Integrit�tsringe, $f:A\longrightarrow B$ ein \rhom\ und $S\subset B$ eine multiplikativ abgeschlossene Teilmenge. Setze \mbox{$T:=f^{-1}(S)$}, dann gilt
\[
\Omega_{B_S/A}\cong\Omega_{B_S/A_T}
\]
\end{lem}
\bew
Jede $A_T$-\deri\ auf $B_S$ ist insbesondere eine $A$-\deri. Umgekehrt ist auch jede $A$-\deri\ auf $B_S$ eine $A_T$-\deri, denn sei $v:B_S\longrightarrow M$ eine $A$-\deri\ in einen $B_S$-Modul $M$, und sei $\frac at\in A_T$, d.h. $a\in A$ und $t\in T\subset A$. Es gilt
\[0=\frac 1t\cdot v(1)=\frac 1t\cdot v\left(\frac 1 t\cdot t\right)=\frac 1t\left(\frac 1t\cdot v(t)+tv\left(\frac 1 t\right)\right)=v\left(\frac 1 t\right)\]
Aus der Produktregel folgt $v\left(\frac at\right)=0$. Damit ist gezeigt, da� $v$ auch eine $A_T$-\deri\ ist.

Wir wissen nun, da� $d_{B_S/A}$ und $d_{B_S/A_T}$ beide universelle $A_T$-\deri en sind. Aus der universellen Eigenschaft folgt damit die Isomorphie.\sieg
Sei $f:X\longrightarrow Y$ ein endlicher Morphismus zwischen Kurven. Sei $V\cong\Spec A$ eine affine Teilmenge von $Y$, dann ist $U:=f^{-1}(V)$ affin, d.h. $U\cong\Spec B$. Wir erhalten dann eine Garbe $(\Omega_{B/A})^\sim$ von $\O_U$-Moduln auf dem affinen Unterschema $U$.

Nun k�nnen wir die Garbe $\Omega_{X/Y}$ von $\O_X$-Moduln auf $X$ definieren, indem wir $X$ und $Y$ affin �berdecken und die zugeh�rigen Garben $(\Omega_{B/A})^\sim$ zusammenkleben. $\Omega_{X/Y}$ ist insbesondere eine koh�rente Garbe.
\begin{lem}
\label{suppkeinnull}
Seien $A$ und $B$ $K$-Algebren, und sei $f:A\longrightarrow B$ ein modulendlicher $K$-Algebren\homo. Dann ist $B$ verm�ge $f$ eine $A$-Algebra und es gilt $(\Omega_{B/A})_{(0)}=0$.
\end{lem}
\bew
$\Omega_{B/A}$ wird von $\{d_{B/A}(b)\mid b\in B\}$ erzeugt, wir zeigen da� jedes dieser Erzeugenden ein Torsionselement ist, d.h. zu $b\in B$ \ex\ ein $0\not=c\in B$, so da� $c\cdot d_{B/A}(b)=0$ gilt. Wenn das gezeigt ist, folgt die Behauptung, denn es gilt $(\Omega_{B/A})_{(0)}\cong B_{(0)}\otimes_B\Omega_{B/A}$ nach Lemma \ref{diff_supp_hilfe} mit $S:=B\setminus\{0\}$ und Lemma \ref{konstfield}. F�r $1\otimes d_{B/A}(b)$ mit $b\in B$ gilt
\[1\otimes d_{B/A}(b)=\left(\frac 1c\cdot c\right)\otimes d_{B/A}(b)=\frac 1 c\otimes cd_{B/A}(b)=\frac 1 c\otimes 0=0\]
Es bleibt noch zu zeigen, da� $d_{B/A}(b)$ mit $b\in B$ ein Torsionselement ist. Jede modulendliche Ringerweiterung ist eine ganze Ringerweiterung. Sei nun $b\in B$, dann erf�llt $b$ eine Ganzheitsgleichung �ber $A$, d.h. $\sum\limits_{i=0}^na_ib^i=0$ mit $a_i\in A$ und $a_n=1$. Wir nehmen an, da� die Ganzheitsgleichung so gew�hlt ist, da� $n$ minimal ist. Man sieht leicht, da� $d_{B/A}(b^n)=nb^{n-1}d_{B/A}(b)$ gilt. Bei uns haben alle K�rper Charakteristik $0$, also hat $n$ ein Inverses in $B$.
\[0=\frac 1nd_{B/A}\left(\sum\limits_{i=0}^na_ib^i\right)=\frac 1n\sum\limits_{i=1}^na_iib^{i-1}d_{B/A}(b)=\left(\sum\limits_{i=0}^{n-1}a_{i+1}\frac{i+1}nb^i\right)d_{B/A}(b)\]
Aus der Minimalit�t von $n$ und aus $a_n=1$ folgt $c:=\sum\limits_{i=0}^{n-1}a_{i+1}\frac{i+1}nb^i\not=0$. Damit ist gezeigt, da� $d_{B/A}(b)$ ein Torsionselement ist.\sieg
\begin{fol}
\label{diffcritabg}
Seien $X=\Spec B$ und $Y=\Spec A$ affine Kurven �ber $C$, und sei $f:X\longrightarrow Y$ ein endlicher Morphismus. $f$ wird von einem $C$-Algebrenhomomorphismus $A\longrightarrow B$ induziert und es gilt \[\crit(f)=\supp\Omega_{B/A}\]
\end{fol}
Insbesondere ist $\crit(f)$ abgeschlossen in $\Spec B$.\medskip\\
\bew
Nach Lemma \ref{suppkeinnull} gilt $(0)\not\in\supp\Omega_{B/A}$. Nun betrachten wir die abgeschlossenen Punkte.

Sei $\m_B$ ein maximales Ideal in $B$, und sei $\m_A\subset A$ das Bild von $\m_B$ unter $f$. $B$ ist in nat�rlicher Weise eine $A$-Algebra und es gilt nach Satz \ref{diff_supp_hilfe} mit $S:=B\setminus\m_B$ \[(\Omega_{B/A})_{\m_B}\cong\Omega_{B_{\m_B}/A}\]
$B_{\m_B}$ ist eine $A_{\m_A}$-Algebra verm�ge dem induzierten lokalen \rhom. Nach Lemma \ref{konstfield} gilt
\[
\Omega_{B_{\m_B}/A}\cong\Omega_{B_{\m_B}/A_{\m_A}}
\]
insgesamt erhalten wir dann
\[
(\Omega_{B/A})_{\m_B}\cong\Omega_{B_{\m_B}/A_{\m_A}}
\]
Aus Satz \ref{diff_supp_lokal} folgt damit die Behauptung.\sieg
\begin{fol}
\label{endlkritendl}
Sei $f:X\longrightarrow Y$ ein endlicher Morphismus zwischen Kurven. Dann ist die Menge der kritischen Punkte von $f$ abgeschlossen und damit endlich.
\end{fol}
Denn: Sei $V\subset Y$ affin, dann ist auch $U:=f^{-1}(V)$ affin. Die Menge der kritischen Punkte von $f_{|U}$ ist dann nach Folgerung \ref{diffcritabg} abgeschlossen in $U$. Nun ist Angeschlossenheit eine lokale Eigenschaft, damit ist die Menge der kritischen Punkte von $f$ abgeschlossen.

Der generische Punkt ist kein kritischer Punkt, also ist die Menge der kritischen Punkte von $f$ endlich.\sieg

\section{Die projektive Gerade}
Im Folgenden ist $C$ ein algebraisch abgeschlossener K�rper. In diesem Abschnitt werden einige Eigenschaften des projektiven Schemas $\Pc$, die wir sp�ter ben�tigen werden, betrachtet.

Wir betrachten den Polynomring in zwei Ver�nderlichen $C[X,Y]$. Sei\\ $S_+:=(X,Y)$ das von $X$ und $Y$ erzeugte Ideal. Die zugrundeliegende Menge von
\[\P1:=\proj C[X,Y]\]
besteht aus denjenigen homogenen Primidealen $\p\in C[X,Y]$, die echt in $S_+$ enthalten sind. Dazu geh�rt der nicht-abgeschlossene Punkt $(0)$ und die abgeschlossenen Punkte $\p\in\Pc$ von der Form $\p=(aX+bY)$ mit $(a,b)\in C^2\setminus\{(0,0)\}$.

Die abgeschlossenen Mengen in $\P1$ sind Teilmengen von der Form \\$V(I)=\{\p\in\Pc\mid I\subset\p\}$ mit einem homogenen Ideal $I\subset C[X,Y]$.
\subsection{Polynome als Morphismen}
\label{abschnitt_polymor}
Im Folgenden werden wir sehen, wie man ein nicht-konstantes Polynom\\ $f\in C[X]$ als Morphismus $\Pc\longrightarrow\Pc$ auffassen kann. Zun�chst betrachten wir den einfacheren Fall, indem wir zu einem gegebenen Polynom $f$ einen Morphismus $\Spec C[X]\longrightarrow\Spec C[X]$ konstruieren. Anschlie�end werden wir diese Konstruktion auf $\Pc$ erweitern.

Sei $f\in C[X]$ ein Polynom. Wir definieren einen $C$-Algebren\homo
\begin{eqnarray*}
f^*:C[X]&\longrightarrow&C[X]\\
X&\longmapsto&f
\end{eqnarray*}
Dieser induziert einen Morphismus
\[\Spec f^*:\Spec C[X]\longrightarrow\Spec C[X]\]
Sei $(X-\lambda)\in\Spec C[X]$ ein abgeschlossener Punkt, dann gilt
\[\Spec f^*((X-\lambda))=(X-\mu)\textrm{ mit }\mu=f(\lambda)\]
Denn: $\lambda$ eine Nullstelle von $f-\mu$, also gilt $f-\mu\in(X-\lambda)$. $f-\mu$ ist das Bild von $X-\mu$ unter $f^*$, also umfa�t das Urbild vom Ideal $(X-\lambda)$ das Ideal $(X-\mu)$. Aus der Maximalit�t von $(X-\mu)$ folgt die Gleichheit. Man sieht genauso leicht, da� der generische Punkt $(0)$ unter $\Spec f^*$ auf selbigen abgebildet wird.
% falls $f$ nicht konstant ist.

Wir k�nnen also die Punkte aus $C$ mit den abgeschlossenen Punkten aus $\Spec C[X]$ identifizieren, so da� diese Identifikation mit $f\longmapsto\Spec f^*$ vertr�glich ist.
Mit dem folgenden Lemma erhalten wir eine Kennzeichung der kritischen Punkte von $\Spec f^*$. F�r den Beweis wird eine Bemerkung aus Kapitel \ref{kapitel_verzw} ben�tigt.
%Q Bosch 3.6 sep KE Lemma 1
\begin{lem}
Sei $(X-\lambda)\in\Spec C[X]$ ein abgeschlossener Punkt, dann ist $(X-\lambda)$ genau dann ein kritischer Punkt von $\Spec f^*$, wenn $f'(\lambda)=0$ gilt.
\end{lem}
\bew
Setze $\mu:=f(\lambda)$, dann ist $\lambda$ eine Nullstelle von $f-\mu$, d.h. $f-\mu\in(X-\lambda)$. Es gilt $f-\mu\in(X-\lambda)^2$ genau dann, wenn $\lambda$ eine mehrfache Nullstelle von $f-\mu$ ist. Nun ist allgemein bekannt, da� $\lambda$ genau dann eine mehrfache Nullstelle von $f-\mu$ ist, wenn $f'(\lambda)=(f-\mu)'(\lambda)=0$ gilt.

 $\Spec f^*$ induziert einen lokalen \rhom\
\begin{eqnarray*}
\varphi:C[X]_{(X-\mu)}&\longrightarrow&C[X]_{(X-\lambda)}\\
X&\longmapsto&f
\end{eqnarray*}
zwischen nach maximalen Idealen lokalisierten Polynomringen. $X-\mu$ und $X-\lambda$ sind die Erzeugenden der entsprechenden maximalen Ideale.

Nun gilt $f'(\lambda)=0$ genau dann, wenn $X-\mu$ unter $\varphi$ nach $(X-\lambda)^2$ abgebildet wird. Nach Bemerkung \ref{verzequivkrit} ist da� �quivalent dazu, da� $(X-\lambda)$ ein kritischer Punkt von $\Spec f^*$ ist.\sieg
Nun werden wir $f$ einen Morphismus $\Pc\longrightarrow\Pc$ zuordnen. Sei $n$ der Grad von $f(X)$, und sei $F(X,Y):=Y^nf\left(\frac XY\right)$ die Homogenisierung nach $Y$. Es gilt $n\geq 1$, da $f$ nicht-konstant ist. Wir definieren einen $C$-Algebren\homo
\begin{eqnarray*}
F^*:C[X,Y]&\longrightarrow&C[X,Y]\\
X&\longmapsto&F\\
Y&\longmapsto&Y^n
\end{eqnarray*}
$F^*$ ist injektiv, und homogene Polynome werden unter $F^*$ auf homogene Polynome abgebildet. Umgekehrt sind auch die Urbilder homogener Polynome homogen. Wir betrachten nun die Urbilder homogener Primideale. Das Urbild von $(Y)$ ist $(Y)$. Wegen der Injektivit�t von von $F^*$ ist das Urbild von $(0)$ ebenfalls $(0)$. Es fehlt noch das Urbild von $(X-\lambda Y)$ mit $\lambda\in C$.

Setze $\mu:=f(\lambda)$, dann ist $X-\lambda$ ein Teiler von $f-\mu$, d.h. es \ex\ ein $g\in C[X]$ mit
\[f-\mu=g\cdot (X-\lambda)\]
Wir homogenisieren beide Seiten der Gleichung nun nach $Y$:
\begin{eqnarray*}
Y^n\cdot\left(f\left(\frac XY\right)-\mu\right)&=&Y^n\cdot g\left(\frac XY\right)\cdot\left(\frac XY-\lambda\right)\\
\Longleftrightarrow F(X,Y)-\mu Y^n&=&Y^{n-1}\cdot g\left(\frac XY\right)\cdot (X-\lambda Y)
\end{eqnarray*}
Also ist $X-\lambda Y$ ein Teiler von $F(X,Y)-\mu Y^n$, damit gilt
\[F(X,Y)-\mu Y^n\in(X-\lambda Y)\]
Damit ist $(X-\mu Y)$ im Urbild von $(X-\lambda Y)$ enthalten. Nun ist $(X-\mu Y)$ ein maximales homogenes Ideal, daraus folgt die Gleichheit.

Wir erhalten damit eine stetige Abbildung
\[\proj F^*:\proj C[X,Y]\longrightarrow\proj C[X,Y]\]

Als n�chstes werden wir die kritischen Punkte von $\proj F^*$ kennzeichnen. Wir betrachten das affine Unterschema $D_+(Y)\subset\Pc$, das sind alle Punkte bis auf den Punkt im unendlichen $(Y)$. Es gilt $\Spec C[X]\cong D_+(Y)$, und der Morphismus ${\proj F^*}_{|D_+(Y)}$ entspricht dem Morphismus $\Spec f^*$. Wir erhalten folgendes Resultat:
\begin{fol}
$(X-\lambda Y)$ ist genau dann ein kritischer Punkt von $\proj F^*$, wenn $f'(\lambda)=0$ gilt.
\end{fol}
\subsection{Gebrochen-lineare  \Tf en als Morphismen}
\label{abschnitt_gbmor}
Wir betrachten bijektive \gb e \Tf en $\frac{ax+b}{cx+d}$, d.h. es gilt
\[\det \left[\begin{array}{cc}
a&b\\c&d
\end{array}
\right]\not=0
\]
Zu einer gegebenen \gb en \Tf\ definieren wir einen $C$-Algebren\homo
\begin{eqnarray*}
C[X,Y]&\longrightarrow&C[X,Y]\\
X&\longmapsto&aX+bY\\
Y&\longmapsto&cX+dY
\end{eqnarray*}
Nun betrachten wir die Urbilder von Punkten aus $D_+(Y)$, also Punkte der Form $(X-\lambda Y)$ mit $\lambda\in C$. Dazu unterscheiden wir zwei F�lle:
\begin{description}
\item[$\lambda\not=-\frac dc$:]\ \\
Das Urbild von $(X-\lambda Y)$ ist das Ideal $(X-\mu Y)$ mit $\mu=\frac{a\lambda+b}{c\lambda+d}$. Denn: Das Bild von $X-\mu Y$ ist $(aX+bY)-\mu(cX+dY)$. Wir multiplizieren mit der Einheit $c\lambda+d$:
\begin{eqnarray*}
&&(c\lambda+d)(aX+bY)-(a\lambda+b)(cX+dY)\\
&=&(ac\lambda+ad-ac\lambda-bc)X+(bc\lambda+bd-ad\lambda-bd)Y\\
&=&(ad-bc)X+(bc\lambda-ad\lambda)Y\\
&=&(ad-bc)X-\lambda(ad-bc)Y
\end{eqnarray*}
Wir dividieren nochmal durch $ad-bc\not=0$, dann wird $X-\lambda Y$ getroffen.
\item[$\lambda=-\frac dc$:]\ \\
Das Urbild von $(X-\lambda Y)=(cX+dY)$ ist das Ideal $(Y)$.
\end{description}
Wir k�nnen \gb e \Tf en also als Morphismen\\ $\Pc\longrightarrow\Pc$ betrachten. Diese sind sogar Isomorphismen und haben deshalb keine kritischen Punkte.
\begin{satz}
\label{gbdreisatz}
Seien $x_1,x_2,x_3\in\Pc$ paarweise verschiedene abgeschlossene Punkte. Dann \ex\ eine \gb e \Tf\ $q$ mit \[x_1\longmapsto 0,\ x_2\longmapsto 1,\ x_3\longmapsto\infty\]
\end{satz}
\bew
Setze $$q(x):=\frac{x-x_1}{x-x_3}\cdot\frac{x_2-x_3}{x_2-x_1}$$
$q$ ist bijektiv, da $\det \left[\begin{array}{cc}
1&x_1\\1&x_3
\end{array}
\right]\not=0
$ gilt.\sieg
\subsection{$K$-rationale Punkte}
Sei $\sigma\in\Aut(C)$, wir definieren einen \rhom
\begin{eqnarray*}
C[X,Y]&\longrightarrow&C[X,Y]\\
\sum a_{ij}X^iY^j&\longmapsto&\sum \sigma(a_{ij})X^iY^j
\end{eqnarray*}
indem wir auf den Koeffizienten der Polynome operieren. Das Urbild vom Punkt $(aX+bY)\in\Pc$ ist $(\sigma^{-1}(a)X+\sigma^{-1}(b)Y)$. Wir erhalten einen Isomorphismus
\[\proj(\sigma):(\Pc)^\sigma\longrightarrow\Pc\]
Das folgende Diagramm kommutiert:
\[\xymatrix{
(\Pc)^\sigma\ar[rr]^{\proj(\sigma)}\ar[d]&&\Pc\ar[d]\\
C^\sigma\ar[rr]_{\Spec(\sigma)}&&C
}\]
Also ist die $C$-\var\ $(\Pc)^\sigma$ isomorph zur $C$-\var\ $\Pc$. Der Funktionenk�rper $K(\Pc)$ ist isomorph zum K�rper der rationalen Funktionen in einer Ver�nderlichen, also $C(X)$. Der durch $\proj(\sigma)$ induzierte K�rperautomorphismus ist
\begin{eqnarray*}
\proj(\sigma)^*:C(X)&\longrightarrow&C(X)\\
a\in C&\longmapsto&\sigma(a)\\
X&\longmapsto&X
\end{eqnarray*}
\begin{df}
Sei $K\subset C$ ein Unterk�rper, dann hei�t ein abgeschlossener Punkt $(aX+bY)\in\Pc$ $K$-rational, falls ein $\lambda\in C$ \ex, so da� $\lambda a,\lambda b\in K$ gilt.
\end{df}
\begin{bem}
\label{qratinv}
F�r $\sigma\in\Aut(C:K)$ sind $K$-rationale Punkte invariant unter $\proj(\sigma)$.
\end{bem}
Denn: sei $(aX+bY)\in\Pc$ ein $K$-rationaler Punkt, d.h. es \ex\ ein $\lambda\in C$ mit $\lambda a,\lambda b\in K$. Es gilt
\[
\lambda a\in K\Longrightarrow\sigma^{-1}(\lambda a)\in K\Longrightarrow\sigma^{-1}(\lambda)\sigma^{-1}(a)\in K
\]
Analog gilt $\sigma^{-1}(\lambda)\sigma^{-1}(b)\in K$, also ist auch $(\sigma^{-1}(a)X+\sigma^{-1}(b)Y)$ $K$-rational.
\begin{bem}
\label{lok_inv_rat}
Sei $(aX+bY)\in\Pc$ ein $K$-rationaler Punkt, dann \ex\ im lokalen Ring $C[X,Y]_{((aX+bY))}$ ein lokaler Parameter, der invariant unter $\Aut(C:K)$ ist.
\end{bem}

\section{Galois �berlegungen}
Im Folgenden ist $C$ ein algebraisch abgeschlossener K�rper und alle auftretenden K�rper haben Charakteristik 0.
\subsection{Galois Korrespondenz}
\begin{satz}[Fortsetzungssatz]
\label{fortsatzalg2}
Sei $K$ ein K�rper, $C$ ein algebraisch abge\-schlos\-sener K�rper und $\sigma:K\longrightarrow C$ ein \khom. F�r jede algebra\-ische Erweiterung $L$ von $K$ existiert eine Fortsetzung zu einem \khom\ $\widetilde\sigma:L\longrightarrow C$.
\end{satz}
Beweis mit Zornschem Lemma.\sieg
%Q Belyi's Theorem Revisited Lemma 1.4
%Q Bosch, Algebra Transzendenzbasen
%N Die erste Aussage kann ich f�r \kga gebrauchen
%N Fortsetzungssatz wird benutzt, siehe Algebra2 S.11
%N AlgGeo 1 S.5 ist besser. brauchen K[T]->K[T] injektiv, dann �ber Quotientenk�rper argumentieren
\begin{lem}
\label{auto_fortsetung}
Sei $K\subset C$ ein Unterk�rper, dann kann jeder K�rperautomorphismus $\sigma\in\Aut(K)$ zu einem Automorphismus auf ganz $C$ fortgesetzt werden.
\end{lem}
\bew
Sei $\tau$ eine Transzendenzbasis von $C:K$. Nun l��t sich jedes $x\in K(\tau)$ als rationale Funktion in $x_1,\dots,x_s\in\tau$ mit Koeffizienten aus $K$ auffassen. Also k�nnen wir einen K�rperautomorphismus $\widetilde\sigma:K(\tau)\longrightarrow K(\tau)$ definieren, wobei $\sigma$ f�r jedes $x\in K(\tau)$ auf den Koeffizienten operiert. Es gilt $\widetilde\sigma_{|K}=\sigma$ und $\widetilde\sigma_{|\tau}=\id_\tau$, also ist $\widetilde\sigma$ eine Fortsetzung von $\sigma$ auf $K(\tau)$. Als n�chstes m�ssen wir $\widetilde\sigma$ zu einem Automorphismus auf $C$ fortsetzen. $C:K(\tau)$ ist eine algebraische Erweiterung und $\widetilde\sigma$ ist ein K�rperhomomorphismus von $K(\tau)$ nach $C$. Nach Satz \ref{fortsatzalg2} existiert eine Fortsetzung $\overline\sigma:C\longrightarrow C$ von $\widetilde\sigma$. Das Bild von $\overline\sigma$ ist ein algebraisch abgeschlossener Unterk�rper von $C$, da $C$ algebraisch abgeschlossen ist. Also gilt $\Im\overline\sigma=C$, damit ist $\overline\sigma$ ein K�rperautomorphismus.\sieg
%N F�R K�CK 3.2
Anwendung von Lemma \ref{auto_fortsetung}:
\begin{satz}
\label{lemma_vrf}
Sei $K\subset C$ Unterk�rper, dann gilt
\[C^{\Aut(C:K)}=K\]
\end{satz}
Beweis:
\begin{description}
\item [,,$\subset$'']\ \\
Sei $x\in C\smallsetminus K$. Es ist zu zeigen, da� ein $\sigma\in\Aut(C:K)$ \ex, so da� $\sigma(x)\not=x$ gilt.
\begin{description}
\item[\rm Fall 1:]$x$ ist transzendent �ber $K$\\
W�hle $\sigma\in\Aut(K(x):K)$ mit $\sigma(x)=-x$. $\sigma$ kann nach Lemma \ref{auto_fortsetung} zu einem Automorphismus auf ganz $C$ fortgesetzt werden.
\item[\rm Fall 2:]\ \\
%N Aufjedenfall ist K(x) isomorph zu K(y). Vielleicht geht das auch ohne normale H�lle
$x$ hat ein Minimalpolynom $m_x\in K[X]$. Sei
\[N:=\{x\in C\mid m_x(x)=0\}\]
die Nullstellenmenge von $m_x$, dann ist $K(N)$ der Zerf�llungsk�rper von $m_x$. $m_x$ ist separabel, also k�nnen wir ein $\sigma\in\Aut(K(N):K)$ mit \mbox{$\sigma(x)\not=x$} w�hlen. $\sigma$ kann zu einem Automorphismus auf ganz $C$ fortgesetzt werden.
\end{description}
\item [,,$\supset$'']\ \\
klar per Definition
\end{description}
\sieg
%Q Belyi's Theorem Revisited
\begin{df}
Eine Untergruppe $U\subset\Aut(C)$ hei�t abgeschlossen, falls ein Unterk�rper $K\subset C$ mit $U=\Aut(C:K)$ \ex.
\end{df}
\begin{bem}
$U\subset\Aut(C)$ abgeschlossen $\impl$ $U=\Aut(C:C^U)$
\end{bem}
\bew
Sei $U$ abgeschlossen, d.h. $U=\Aut(C:L)$ mit einem Unterk�rper $L\subset C$. Nach Satz \ref{lemma_vrf} gilt $C^U=L$ und daraus folgt $U=\Aut(C:C^U)$.\sieg
Wir erhalten damit eine Galois Korrespondenz zwischen den abgeschlossenen Untergruppen von $\Aut(C)$ und den Unterk�rpern von $C$:
\begin{eqnarray*}
K&\longmapsto&\Aut(C:K)\\
U&\longmapsto&C^U
\end{eqnarray*}
\subsection{Unendliche \KE en}
%Q Belyi's Theorem Revisited Lemma 1.5
%Q Alebra 2 S. 25
%N K^phi(U)=C^U steht auf nem Schmierzettel, wird f�r Lemma 1.5 und 1.6 benutzt
%N VERBESSERN! NACH GAMST IST DAS EINE NB
\begin{lem}
\label{abgkriteri}
Sei $U\subset\Aut(C)$ eine Untergruppe und $K:C^U$ eine endliche K�rpererweiterung mit $\Aut(C:K)\subset U$. Dann ist $U$ abgeschlossen.
\end{lem}
\bew
Wenn wir $K$ durch seinen normalen Abschlu� ersetzen, gelten die Voraussetzungen weiterhin, also k�nnen wir annehmen, da� $K:C^U$ eine normale und damit galoissche \KE\ ist. Wir haben nun den Einschr�nkungshomomorphismus
\begin{eqnarray*}
\varphi:\Aut(C:C^U)&\longrightarrow&\Aut(K:C^U)\\
\sigma&\longmapsto&\sigma_{|K}
\end{eqnarray*}
Per Definition gilt $U\subset\Aut(C:C^U)$. Sei $\varphi(U)$ das Bild von $U$, dann gilt $K^{\varphi(U)}=C^U$ und daraus folgt $\varphi(U)=\Aut(K:C^U)$, da die \KE\ $K:C^U$ galoissch und endlich ist.

Es gilt $\ker\varphi=\Aut(C:K)$ und $\Aut(C:K)\subset U\subset\Aut(C:C^U)$, also erhalten wir einen Isomorphismus
\[\Aut(C:C^U)/\Aut(C:K)\cong\Aut(K:C^U)\]
und damit gilt $U=\Aut(C:C^U)$.\sieg
%Q Algebra 2 S.41
\begin{bem}
\label{normal_zusatz}
Sei $U\subset\Aut(C)$ eine Untergruppe, und sei $\sigma\in\Aut(C)$ ein K�rperautomorphismus. Dann gilt
\[C^{\sigma U\sigma^{-1}}=\sigma(C^U)\]
\end{bem}
Beweis:
\begin{eqnarray*}
x\in C^{\sigma U\sigma^-1}&\Longleftrightarrow&\forall\tau\in U\colon x=\sigma\tau\sigma^{-1}(x)\\
&\Longleftrightarrow&\forall\tau\in U\colon\sigma^{-1}(x)=\tau(\sigma^{-1}(x))\\
&\Longleftrightarrow&\sigma^{-1}(x)\in C^U\\
&\Longleftrightarrow&x\in\sigma(C^U)
\end{eqnarray*}
\sieg
%N F�R BEWEIS VON 66
\begin{lem}
\label{normal_endlich_easy2see}
Sei $G$ eine Gruppe, $H\subset G$ eine Untergruppe von endlichem Index. Dann \ex\ eine Untergruppe $N\subset G$, die normal und von endlichem Index ist, so da� $N\subset H$ gilt.
\end{lem}
\bew
Zuerst zeigen wir, da� f�r Untergruppen $H_1$, $H_2$ von endlichem Index die Untergruppe $H_1\cap H_2$ endlichen Index hat.  Sei $a\in G$, dann gilt \\$a(H_1\cap H_2)=aH_1\cap aH_2$. Nun haben $H_1$, $H_2$ nur endlich viele Nebenklassen, daraus folgt, da� $H_1\cap H_2$ endlich viele Nebenklassen hat. Wir betrachten f�r alle $\sigma\in G$ die konjugierten $\sigma H\sigma^{-1}$ von $H$, dann folgt induktiv, da�
\[N:=\bigcap\limits_{\sigma\in G}\sigma H\sigma^{-1}\]
endlichen Index in $G$ hat. Es bleibt zu zeigen, da� $N$ normal in $G$ ist. F�r $\tau\in G$ gilt
\[
\tau N\tau^{-1}=\tau\left(\bigcap\limits_{\sigma\in G}\sigma H\sigma^{-1}\right)\tau^{-1}=\bigcap\limits_{\sigma\in G}(\tau\sigma)H(\tau\sigma)^{-1}=\bigcap\limits_{\sigma\in G}\sigma H\sigma^{-1}=N
\]
\sieg
%Q Bosch S.140
\begin{satz}
\label{lemma_weildescent_endlerz}
Sei $L$ ein K�rper, $G\subset\Aut(L)$ eine endliche Untergruppe, und sei $K:=L^G$ der Fixk�rper von $G$. Dann ist die \KE\ $L:K$ endlich und  galoissch.
\end{satz}
Beweis: Satz 4 auf S.140 in \cite{bosch}
%Q K�ck Lemma 1.6
%Q Bosch S.18
\begin{lem}
\label{indendldegendl}
Sei $U\subset\Aut(C)$ eine Untergruppe und $V\subset U$ eine Untergruppe von endlichem Index. Dann ist die \KE\ $C^V:C^U$ endlich algebraisch. Falls $V$ normal in $U$ oder $U$ abgeschlossen ist, gilt
\[
[C^V:C^U]\leq[U:V]
\]
\end{lem}
\bew
Nach Lemma \ref{normal_endlich_easy2see} \ex\ eine Untergruppe $W\subset U$, die normal und von endlichem Index ist, so da� $W\subset V$ gilt. Wegen Bemerkung \ref{normal_zusatz} gilt $\sigma(C^W)=C^W$ f�r alle $\sigma\in U$. Nun k�nnen wir folgenden \ghom\ definieren:
\begin{eqnarray*}
\varphi:U&\longrightarrow&\Aut(C^W:C^U)\\
\sigma&\longmapsto&\sigma_{|C^W}
\end{eqnarray*}
Es gilt $W\subset\ker\varphi$, damit erhalten wir kanonisch
\begin{eqnarray*}
\widetilde\varphi:U/W&\longrightarrow&\Aut(C^W:C^U)
\end{eqnarray*}
Nun gilt $(C^W)^{\varphi(U)}=C^U$, und aus $U/W$ endlich folgt $\varphi(U)$ endlich. Also ist $C^U\subset C^W$ Fixk�rper einer endlichen Gruppe, damit ist die \KE\ $C^W:C^U$ nach Satz \ref{lemma_weildescent_endlerz} endlich und galoissch. Daraus folgt \\$\varphi(U)=\Aut(C^W:C^U)$.
Insbesondere ist $C^V:C^U$ endlich, da $C^V$ ein Zwischenk�rper von $C^W:C^U$ ist. $\widetilde\varphi$ ist surjektiv, also gilt
\[
[C^W:C^U]=\#\Aut(C^W:C^U)\leq\#U/W=[U:W]
\]
Damit gilt $[C^V:C^U]\leq[U:V]$, falls $V$ bereits normal in $U$ ist.

$W$ ist normal und von endlichem Index in $V$, also ist auch $[C^W:C^V]$ mit dem gleichen Argument wie oben endlich und galoissch. Nun gilt allgemein:
\[
[C^V:C^U]=\frac{[C^W:C^U]}{[C^W:C^V]}=\frac{\#\Aut(C^W:C^U)}{\#\Aut(C^W:C^V)}=\#(\Aut(C:C^U)/\Aut(C:C^V))
\]
Sei $U$ abgeschlossen, dann gilt $U=\Aut(C:C^U)$ und $V\subset\Aut(C:C^V)$. Daraus folgt $[C^V:C^U]\leq\#U/V=[U:V]$.\sieg
\subsection{Anhang: Untergruppen von festem Index}
\begin{bem}
\label{stabilisatormaster}
Sei $G$ eine Gruppe und $X$ eine Menge, auf der $G$ operiere. Sei $x_0\in X$, dann erhalten wir eine bijektive Abbildung
\[
\varphi_{x_0}:G/\stab(x_0)\longrightarrow Gx_0
\]
Insbesondere ist der Orbit $Gx_0$ von $x_0$ genau dann endlich, wenn $\stab(x_0)$ endlichen Index in $G$ hat.
\end{bem}
\begin{satz}
\label{hall_freie_gruppe}
Sei $G$ eine endlich erzeugte Gruppe. Sei $d\in\nat\setminus\{0\}$, dann ist die Anzahl der Untergruppen vom Index $d$ endlich.
\end{satz}
\bew
Sei $U\subset G$ eine Untergruppe, dann erhalten wir eine Gruppenoperation
\begin{eqnarray*}
G\times G/U&\longrightarrow&G/U\\
(g,xU)&\longmapsto&(gx)U
\end{eqnarray*}
auf den Restklassen von $G/U$. Die Operation ist transitiv und der Stabilisator der Restklasse $U$ ist $U$ selbst. $G$ operiert nun auf einer endlichen Menge, wir numerieren die Restklassen durch, wobei wir $U$ die $1$ zuordnen. $G$ operiert dann transitiv auf $\{1,\dots,d\}$, wobei $\stab(1)=U$ gilt. Wir erhalten also eine surjektive Zuordnung, wenn wir jeder transitiven Operation von $G$ auf $\{1,\dots,d\}$ die Untergruppe $\stab(1)$ vom Index $d$ zuordnen.

Sei $F$ eine freie Gruppe vom Rang $r$, dann ist ein \ghom\ $F\longrightarrow S_d$, wobei $S_d$ die Permutationsgruppe einer $d$-elementigen Menge bezeichnet, durch die Bilder der freien Erzeugenden eindeutig bestimmt, d.h. es gilt \[\#\hom(F,S_d)=(d!)^r\]

Eine Operation von $G$ auf $\{1,\dots,d\}$ ist das gleiche wie ein \ghom\ $G\longrightarrow S_d$. Nun ist $G$ endlich erzeugt, also gilt
\[\#\hom(G,S_d)\leq(d!)^r\]
\sieg
\begin{satz}
\label{formelhall}
Sei $G$ eine freie Gruppe vom Rang $r$. Sei $d\in\nat\setminus\{0\}$, dann ist die Anzahl $M_{r,d}$ der Untergruppen vom Index $d$ durch folgende Rekursionsformel gegeben:
\begin{eqnarray*}
N_{r,1}&=&1\\
N_{r,d}&=&d(d!)^{r-1}-\sum\limits_{i=1}^{d-1}(d-i)!^{r-1}N_{r,i}
\end{eqnarray*}
\end{satz}
Beweis: Theorem 7.2.9 auf S.105 in \cite{Hall}

\def\gka{gegen Konjugationen abgeschlossen}
\section{Belyi Verfahren}
\label{abschnitt_belyverf}
In diesem Abschnitt werden wir zu einer Kurve $X$ �ber $C$, die bereits �ber $\alg$ definiert ist, einen endlichen Morphismus $t:X\longrightarrow\Pc$ konstruieren, so da� dessen kritische Werte in $\dreier$ liegen. Damit w�re f�r $C=\C$ eine Richtung des Satzes von Belyi bewiesen.

Sei $X$ eine Kurve �ber $C$, die bereits �ber $\alg$ definiert ist, dann \ex\ nach Lemma \ref{koeck34} ein endlicher Morphismus $t_1:X\longrightarrow\Pc$, dessen kritische Werte $\alg$-rational sind. Sei $S_1$ die Menge der kritischen Werte von $t_1$, diese ist nach Folgerung \ref{endlkritendl} endlich. Nach Lemma \ref{koeck35} \ex\ ein endlicher Morphismus $t_2:\Pc\longrightarrow\Pc$, der jeden Punkt aus $S_1$ auf einen $\Q$-rationalen Punkt abbildet und dessen kritische Werte $\Q$-rational sind. Sei nun $S_2$ die Menge der kritischen Werte von $t_2$, dann \ex\ nach Lemma \ref{koeck36} ein endlicher Morphismus $t_3:\Pc\longrightarrow\Pc$, der $S_2\cup t_2(S_1)$ nach $\dreier$ abbildet und dessen kritische Werte in $\dreier$ liegen. $t:=t_3\circ t_2\circ t_1$ ist schlie�lich der gesuchte Morphismus.
\subsection{Erster Schritt}
Zu einer Kurve $X$ �ber $C$, die bereits �ber einem \algabg en Unterk�rper $N\subset C$ definiert ist, \ex\ ein nicht-konstanter Morphismus \mbox{$t:X\longrightarrow\Pc$}, der �ber $N$ definiert ist. Denn: Es \ex\ eine \var\ $X_N$ �ber $N$, so da� $X=X_N\times_{\Spec N}\Spec C$ gilt. $X_N$ ist dann nach Bemerkung \ref{defkldim} eine Kurve. Wir w�hlen einen nicht-konstanten Morphismus $t_N:X_N\longrightarrow\Palg_N$, dieser induziert einen Morphismus $t:X\longrightarrow\Pc$. Ich erinnere daran, da� ein Morphismus zwischen vollst�ndigen Kurven nach Satz \ref{welchemorphendl} genau dann endlich ist, wenn er nicht-konstant ist. Wir zeigen nun, da� die kritischen Werte von $t$ $N$-rational sind.

F�r das Belyi Verfahren ist der Spezialfall $N=\alg$ interessant.
\begin{lem}
\label{support_teilm_vom_anderen}
Sei $K':K$ eine \KE, $A$ eine $K$-Algebra und $B$ eine $A$-Algebra. Setze $A':=A\otimes_K K'$ und $B':=B\otimes_K K'$.\\ Sei $\pi:\Spec B'\longrightarrow\Spec B$ die durch die Inklusion $B\hookrightarrow B'$ induzierte Projektion, dann gilt
\[
\pi(\supp\Omega_{B'/A'})\subset\supp\Omega_{B/A}
\]
\end{lem}
\bew
Es gilt
\[B\otimes_A A'=B\otimes_A(A\otimes_K K')=(B\otimes_AA)\otimes_K K'=B\otimes_K K'=B'\]
Nach Satz \ref{diff_supp_hilfe} gilt $\Omega_{B'/A'}\cong\Omega_{B/A}\otimes_B B'$. Sei $\p\in\Spec B'$ ein abgeschlossener Punkt, dann induziert $\pi$ einen lokalen \rhom\ $\varphi: B_{\pi(\p)}\longrightarrow B'_\p$. $B'_\p$ ist damit ein $B_{\pi(\p)}$-Modul, also gilt $B'_\p=B_{\pi(\p)}\otimes_{B_{\pi(\p)}}B'_\p$.
\begin{eqnarray*}
(\Omega_{B'/A'})_\p&\cong&\Omega_{B'/A'}\otimes_{B'}B'_\p\\
&\cong&(\Omega_{B/A}\otimes_BB')\otimes_{B'}B'_\p\\
&\cong&\Omega_{B/A}\otimes_B(B'\otimes_{B'}B'_\p)\\
&\cong&\Omega_{B/A}\otimes_BB'_\p\\
&\cong&\Omega_{B/A}\otimes_B(B_{\pi(\p)}\otimes_{B_{\pi(\p)}}B'_\p)\\
&\cong&(\Omega_{B/A}\otimes_BB_{\pi(\p)})\otimes_{B_{\pi(\p)}}B'_\p\\
&\cong&(\Omega_{B/A})_{\pi(\p)}\otimes_{B_{\pi(\p)}}B'_\p
\end{eqnarray*}
Aus $(\Omega_{B'/A'})_\p\not=0$ folgt damit $(\Omega_{B/A})_{\pi(\p)}\not=0$, d.h.
\[\p\in\supp \Omega_{B'/A'}\Longrightarrow\pi(\p)\in\supp\Omega_{B/A}\]
\sieg
\begin{lem}
\label{nnrataufnull}
Sei $N\subset C$ ein \algabg er Unterk�rper. Es gilt $C\otimes_NN[X]=C[X]$ und die durch $N[X]\hookrightarrow C[X]$ induzierte Projektion $\Spec C[X]\longrightarrow\Spec N[X]$ bildet alle nicht $N$-rationalen Punkte auf\\ \mbox{$(0)\in\Spec N[X]$} ab.
\end{lem}
\bew
Sei $(X-\lambda)\in\Spec C[X]$ ein nicht $N$-rationaler Punkt, d.h. $\lambda\not\in N$ und damit ist $\lambda$ transzendent �ber $N$. Dann ist $X-\lambda$ f�r alle $0\not=p\in N[X]$ kein Teiler von $p$. Also wird unter der Inklusion $N[X]\hookrightarrow C[X]$ kein Element aus $(X-\lambda)$ au�er der Null getroffen.\sieg
\begin{lem}
\label{koeck34}
Sei $X$ eine Kurve �ber $C$, und sei $t:X\longrightarrow\Pc$ ein endlicher Morphismus. Seien $X$ und $t$ �ber einem \algabg en Unterk�rper $N\subset C$ definiert. Dann sind die kritischen Werte von $t$ $N$-rational.
\end{lem}
\bew
$X$ ist �ber $N$ definiert, d.h. es \ex\ eine $N$-\var\ $X_N$, so da�
\[X=X_N\times_{\Spec N}\Spec C\]
gilt. $\Pc$ ist �ber $N$ definiert, es gilt
\[\Pc\cong\Palg_N\times_{\Spec N}\Spec C\]
Seien $\alpha:X\longrightarrow X_N$ und $\beta:\Pc\longrightarrow\Palg_N$ die kanonischen Projektionen. $t$ ist �ber $N$ definiert, d.h. es \ex\ ein Morphismus $t_N:X_N\longrightarrow\Palg_N$ von $N$-\var en, so da� folgendes Diagramm kommutiert.
\[\xymatrix{
&X_N\ar[d]_{t_N}&&X\ar[ll]_\alpha\ar[d]^t\\
&\Palg_N&&\Pc\ar[ll]^\beta
}\]
Nach Bemerkung \ref{defkldim} hat $X_N$ die gleiche Dimension wie $X$, ist also eindimensional und ist damit eine vollst�ndige Kurve. $t$ ist ein nicht-konstanter Morphismus, damit ist auch $t_N$ nicht-konstant. Also ist $t_N$ ein endlicher Morphismus.

%Q Ha S.88 Step 4
Wir betrachten das affine Unterschema $\Palg_N\setminus\{\infty\}\cong\Spec N[X]$, dann gilt
\[\beta^{-1}(\Palg_N\setminus\{\infty\})=\Pc\setminus\{\infty\}\cong\Spec C[X]\]

$U:=t_N^{-1}(\Palg_N\setminus\{\infty\})$ ist wegen der Endlichkeit von $t_N$ ebenfalls affin, d.h. $U\cong\Spec B$. Das Urbild $U'$ von $U$ unter der Projektion $\alpha$ ist isomorph zu $\Spec B'$ mit $B'=B\otimes_N C$. Wir betrachten nun die kritischen Werte von $t_{|U'}$, denn $\infty\in\Pc$ ist ein $N$-rationaler Punkt und daher brauchen wir ihn nicht zu betrachten. Wir k�nnen uns nun auf affine Teile beschr�nken und erhalten folgendes kommutative Diagramm, wobei wir die Bezeichnungen f�r die Morphismen beibehalten.
\[\xymatrix{
&U\ar[d]_{t_N}&&U'\ar[ll]_\alpha\ar[d]^t\\
&\Palg_N\setminus\{\infty\}&&\Pc\setminus\{\infty\}\ar[ll]^\beta
}\]
Es wird ein kommutives Diagramm von Ringhomomorphismen induziert:
\[\xymatrix{
B\ar[rr]&&B'=B\otimes_N C\\
N[X]\ar[rr]\ar[u]&&C[X]\ar[u]
}\]
Wir haben nun die $C$ bzw. $N$-Algebrenhomomorphismen $C[X]\longrightarrow B'$ und $N[X]\longrightarrow B$. Die Menge der kritischen Punkte von $t$ stimmt nach Folgerung \ref{diffcritabg} mit $\supp\Omega_{B'/C[X]}$ �berein. Nach Lemma \ref{support_teilm_vom_anderen} gilt
\[\alpha(\supp\Omega_{B'/C[X]})\subset\supp\Omega_{B/N[X]}\]
Daraus folgt
\[\supp\Omega_{B'/C[X]}\subset\alpha^{-1}(\supp\Omega_{B/N[X]})\]
Die Menge der kritischen Werte von $t$ stimmt mit $t(\supp\Omega_{B'/C[X]})$ �berein. Es gilt
\[
t(\supp\Omega_{B'/C[X]})\subset t(\alpha^{-1}(\supp\Omega_{B/N[X]}))\subset\beta^{-1}(t_N(\supp\Omega_{B/N[X]}))
\]
F�r die letzte Inklusion brauchen wir die Morphismen nur als Abbildungen zu betrachten. Es gilt $\beta\circ t=t_N\circ\alpha$, insbesondere gilt dann
\begin{eqnarray*}
&&\beta\circ t(\alpha^{-1}(\supp\Omega_{B/N[X]}))=t_N\circ\alpha(\alpha^{-1}(\supp\Omega_{B/N[X]}))\\
&\Longrightarrow&\beta\circ t(\alpha^{-1}(\supp\Omega_{B/N[X]}))\subset t_N(\supp\Omega_{B/N[X]})\\
&\Longrightarrow&\beta^{-1}(\beta\circ t(\alpha^{-1}(\supp\Omega_{B/N[X]})))\subset\beta^{-1}(t_N(\supp\Omega_{B/N[X]}))\\
&\Longrightarrow&t(\alpha^{-1}(\supp\Omega_{B/N[X]}))\subset\beta^{-1}(t_N(\supp\Omega_{B/N[X]}))
\end{eqnarray*}
Damit sind nun alle Inklusionen in der obigen Kette gezeigt. Nach Lemma \ref{suppkeinnull} ist $(0)$ nicht in $\supp\Omega_{B/N[X]}$ enthalten, und aus Lemma \ref{nnrataufnull} folgt damit, da� $\beta^{-1}(t_N(\supp\Omega_{B/N[X]}))$ ausschlie�lich aus $N$-rationalen Punkten besteht.\sieg
\subsection{Zweiter Schritt}
Als n�chstes werden wir zu einer endlichen Teilmenge $S\subset\Pc$ von $\alg$-rationalen Punkten einen nicht-konstanten und damit nach Satz \ref{welchemorphendl} endlichen Morphismus von $\Pc$ nach $\Pc$ konstruieren, der jeden Punkt aus $S$ auf einen $\Q$-rationalen Punkt abbildet und dessen kritische Werte $\Q$-rational sind.
%N Ist die Definition sinnvoll? Ja, laut Algebra2 S.11
\begin{df}
Sei $S\subset\alg$ eine Teilmenge. $S$ hei�t \gka, falls f�r jedes $s\in S$ und jeden Automorphismus $\sigma\in\Aut(\alg/\Q)$ stets $\sigma(s)\in S$ gilt. D.h. $\sigma(S)=S$ f�r alle $\sigma\in\Aut(\alg/\Q)$.
\end{df}
%N x kein Element von Q => Minpoly von x hat Grad > 1 => ... K�rperauto auspacken... fortsetzen auf ganz \alg ... fertig
\begin{bem}
\label{kong_abg_einelem}
Aus $S\subset\alg$ einelementig und \gka\ folgt $S\subset\Q$
\end{bem}
\begin{bem}
\label{kong_abg_bilder}
Sei $p\in\Q[X]$, dann ist $p(S)$ \gka, falls $S$ \gka\ ist.
\end{bem}
\bew
Sei $x\in p(S)$, d.h. es \ex\ $s\in S$ mit $p(s)=x$. Es gilt
\[\sigma(x)=\sigma(p(s))=p(\sigma(s))\]
S ist \gka, also gilt $\sigma(s)\in S$ und daraus folgt $\sigma(x)\in p(S)$.\sieg
%Denn: $\sigma(p(S))=p(\sigma(S))=p(S)$
\begin{bem}
\label{kong_abg_nullst}
Sei $p\in\Q[X]$, dann ist $N(p):=\{x\in\alg\mid p(x)=0\}$ \gka.
\end{bem}
\bew
Sei $x\in\alg$ mit $p(x)=0$ und sei $\sigma\in\Aut(\alg/\Q)$. Es gilt
\[p(\sigma(x))=\sigma(p(x))=\sigma(0)=0\]
\sieg
\begin{bem}
\label{kong_abg_machen}
Sei $S\subset\alg$ eine endliche Teilmenge, dann \ex\ eine endliche Teilmenge $\overline S\subset\alg$ mit $S\subset\overline S$, so da� $\overline S$ \gka\ ist.
\end{bem}
\bew
F�r $s\in S$ bezeichne $m_s\in\Q[X]$ das Minimalpolynom von $s$ �ber $\Q$. Wir definieren ein Polynom
\[p:=\prod\limits_{s\in S}m_s\]
Es gilt $p\in\Q[X]$ und $S\subset N(p)$. Setze $\overline S:=N(p)$, dann ist $\overline S$ nach Bemerkung \ref{kong_abg_nullst} \gka.\sieg
\begin{bem}
\label{kong_abg_expoly}
Sei $S\subset\alg$ endlich und \gka. Setze $n:=\#S$, dann \ex\ ein nicht-konstantes Polynom $p\in\Q[X]$ vom Grad $n$, so da� $p(S)=\{0\}$ gilt.
\end{bem}
\bew
Setze
\[p:=\prod\limits_{s\in S}(X-s)\]
Sei $\sigma\in\Aut(\alg/\Q)$, dann ist $\sigma_{|S}:S\longrightarrow S$ eine Permutation auf $S$, und damit gilt $\sigma(p)=p$. Daraus folgt $p\in\Q[X]$.\sieg
Das folgende Lemma gibt uns den gew�nschten Morphismus, da wir Polynome nach Abschnitt \ref{abschnitt_polymor} als Morphismen von $\Pc$ nach $\Pc$ auffassen k�nnen.
%Q Bely's Theorem Revisited, Lemma 3.5
\begin{lem}
\label{koeck35}
Sei $S\subset\alg$ eine endliche Teilmenge. Es \ex\ ein nicht-konstantes Polynom $p\in\Q[X]$, so da� $p(S)\subset\Q$ gilt und alle kritischen Werte von $p$ in $\Q$ liegen.
\end{lem}
\bew
Nach Bemerkung \ref{kong_abg_machen} k�nnen wir annehmen, da� $S$ \gka\ ist. Induktion nach $n=\#S$:
\begin{description}
\item[$n\leq1$:]\ \\
Setze $p(X):=X$, dann hat $p$ keine kritischen Werte und nach Bemerkung \ref{kong_abg_einelem} gilt $p(S)=S\subset\Q$.
\item[$n>1$:]\ \\
Nach Bemerkung \ref{kong_abg_expoly} \ex\ ein nicht-konstantes Polynom $p_1\in\Q[X]$ vom Grad $n$ mit \mbox{$p_1(S)=\{0\}$}. ${p_1}'$ hat h�chstens $n-1$ Nullstellen. Also hat
\[S_1:=p_1\{r\in\alg\mid {p_1}'(r)=0\}\]
h�chstens $n-1$ Elemente und ist nach Bemerkung \ref{kong_abg_bilder} und \ref{kong_abg_nullst} \gka. Nach \iv\ \ex\ ein nicht-konstantes Polynom $p_2\in\Q[X]$, so da� $p_2(S_1)\subset\Q$ gilt und alle kritischen Werte von $p_2$ in $\Q$ liegen. Setze $p:=p_2\circ p_1$, damit gilt \\$p(S)=p_2\{0\}\subset\Q$. Es bleibt zu zeigen, da� alle kritischen Werte von $p$ in $\Q$ liegen. Es gilt \[p'(X)={p_2}'(p_1(X))\cdot{p_1}'(X)\]
Sei $z$ eine Nullstelle von $p'$, d.h. $p_2(p_1(z))$ ist ein kritischer Wert von $p$.
\begin{description}
\item[\rm Fall 1:]\ \\
${p_1}'(z)=0\impl p_1(z)\in S_1\impl p_2(p_1(z))\in p_2(S_1)\subset\Q$
\item[\rm Fall 2:]\ \\
${p_2}'(p_1(z))=0\impl p_1(z)$ ist ein kritischer Punkt von $p_2\impl p_2(p_1(z))$ ist ein kritischer Wert von $p_2\impl p_2(p_1(z))\in\Q$.
\end{description}
\end{description}
\sieg
\subsection{Dritter Schritt}
Als letztes konstruieren wir einen nicht-konstanten und damit endlichen Morphismus $\Pc\longrightarrow\Pc$, der eine endliche Teilmenge $S\subset\Pc$ von $\Q$-rationalen Punkten nach $\dreier$ abbildet, so da� die kritischen Werte in $\dreier$ liegen.

Da wir Polynome und \gb e \Tf en als Morphismen auffassen k�nnen\footnote{siehe Abschnitt \ref{abschnitt_polymor} und \ref{abschnitt_gbmor}}, k�nnen wir die folgenden Beweise ein wenig einfacher formulieren.
\begin{lem}
\label{lemma_vierer}
Sei $x\in\Pc-\dreier$ ein $\Q$-rationaler Punkt. Dann \ex\ ein nicht-konstanter Morphismus $q$, der $\{0,1,\infty,x\}$ nach $\dreier$ abbildet und dessen kritische Werte in $\dreier$ liegen.
\end{lem}
Als erstes w�hlen wir eine \gb e \Tf\ $q_2$, die $\{0,1,\infty,x\}$ nach $\{0,1,\infty,x'\}$ abbildet, wobei $x'\in(0,1)$ gilt. Wenn $x$ bereits in $(0,1)$ liegt, setzen wir $q_2:=\id$. Falls $x$ dies nicht erf�llt, gibt es zwei F�lle:
\begin{description}
\item[\rm Fall $x<0$:]\ \\
Setze $q_2:=\frac1{1-x}$
\item[\rm Fall $x>1$:]\ \\
Setze $q_2:=\frac1x$
\end{description}
Als n�chstes w�hlen wir ein Polynom $q_1$, das $\{0,1,\infty,x'\}$ nach $\dreier$ abbildet und dessen kritische Werte in $\dreier$ liegen.
Nun gilt $x'\in(0,1)$, also hat $x'$ eine Darstellung $x'=\frac{m}{m+n}$ mit $m,n\in\Nat$. Setze $$q_1(z):=\frac{(m+n)^{m+n}}{m^m n^n}z^m(1-z)^n$$
\newpage
Es gilt:
\begin{enumerate}
\item $q_1(0)=0$
\item $q_1(1)=0$
\item $q_1(\infty)=\infty$
\item $q_1(x')=1$
\end{enumerate}
Die Rechnung zu 4.:
\[{x'}^m(1-x')^n=\left(\frac{m}{m+n}\right)^m\left(\frac{n}{m+n}\right)^n=\frac{m^m n^n}{(m+n)^{m+n}}\]
Nun m�ssen wir noch die kritischen Werte von $q_1$ bestimmen.
\begin{eqnarray*}
q_1'(z)&=&konst\cdot mz^{m-1}(1-z)^n-z^mn(1-z)^{n-1}\\
&=&konst\cdot z^{m-1}(1-z)^{n-1}(m(1-z)-zn)\\
&=&konst\cdot z^{m-1}(1-z)^{n-1}(m-(m+n)z)
\end{eqnarray*}
Als kritische Punkte kommen also $0,1,\infty,x'$ in Frage, damit liegen die kritischen Werte von $q_1$ in $\dreier$.

Den gesuchten Morphismus $q$ erhalten wir schlie�lich durch das Kompositum aus $q_1$ und $q_2$.
\[q:=q_1\circ q_2\]
\sieg
\begin{lem}
\label{koeck36}
Sei $S\subset\Q\cup\{\infty\}$ eine endliche Teilmenge. Es \ex\ ein nicht-konstanter Morphismus $q:\P1\longrightarrow\P1$, so da� $q(S)\subset\{0,1,\infty\}$ und alle kritischen Werte von $q$ in $\{0,1,\infty\}$ liegen.
\end{lem}
Beweis per Induktion nach $n=\#S$
\begin{description}
\item[$n\leq3$:]\ \\
Wir w�hlen eine \gb e \Tf\ $q$, die $S$ nach $\dreier$ abbildet. (siehe Satz \ref{gbdreisatz})
\item[$n>3$:]\ \\
Wir k�nnen annehmen, da� $\dreier\subset S$ gilt. Sei $x\in S\setminus\dreier$ ein weiterer Punkt, dann \ex\ nach Lemma \ref{lemma_vierer} ein nicht-konstanter Morphismus $q_1$, der $\{0,1,\infty,x\}$ nach $\dreier$ abbildet und dessen kritische Werte in $\dreier$ liegen. Es gilt dann $\#q_1(S)<n$. Nach \iv\ \ex\ ein nicht-konstanter Morphismus $q_2$, der $q_1(S)$ nach $\dreier$ abbildet und dessen kritische Werte in $\dreier$ liegen. $q:=q_2\circ q_1$ ist schlie�lich der gesuchte Morphismus.
\end{description}
\sieg

\def\homeo{Hom�omorphi}
\def\cov{�berlagerung}
\def\zus{zusammenh�ngend}
\def\mannig{Mannigfaltigkeit}
\def\zusman{\zus e \mannig}
\def\hol{holomorph}
\def\rf{Riemannsche Fl\"ache}
\section{Vorbereitungen II}
\subsection{\cov stheorie}
Eine \zusman\ ist ein \zus er topologischer Raum, der lokal hom�omorph zum $\mathbbm R^n$ ist, d.h. eine $n$-dimensionale reelle $C^0$-\mannig. Eine \zusman\ ist insbesondere wegweise und lokal wegweise \zus.
% Forster S. 24
\begin{df}
\label{defcov}
Seien $X$ und $Y$ \zusman en. Eine stetige Abbildung $p:Y\longrightarrow X$ hei�t \textbf{\cov}, falls folgendes gilt:\smallskip\\
Jedes $x\in X$ hat eine offene Umgebung $U$, so da� sich $p^{-1}(U)$ als Vereinigung von paarweise disjunkten offenen Teilmengen $V_j\subset X$, wobei $j$ ein Element aus einer Indexmenge $J$ ist, darstellen l��t:
\[
p^{-1}(U)=\bigcup\limits_{j\in J}V_j
\]
Zus�tzlich sind f�r alle $j\in J$ die Einschr�nkungen $p_{|V_j}:V_j\longrightarrow U$ \homeo smen.
\\\emph{Sprechweise:} $Y$ ist eine \cov\ von $X$.
\end{df}
%Q Fo S26 thm 4.16
F�r $y_1,y_2\in Y$ haben $p^{-1}(y_1)$ und $p^{-1}(y_2)$ die gleiche M�chtigkeit. Falls diese endlich ist, d.h. $\#p^{-1}(y_1)=n$ mit $n\in\nat$, bezeichnet $n$ den \textbf{Grad} von $p$. Insbesondere ist eine \cov\ $p:Y\longrightarrow X$ surjektiv, da $Y$ nicht-leer ist.\footnote{siehe Theorem 4.16 auf S.26 in \cite{Forster}}
%Q Fo S.31
\begin{df}
\label{defunivcov}
Seien $X$ und $Y$ \zusman en, und sei $p:Y\longrightarrow X$ eine \cov. Dann hei�t das Paar $(Y,p)$ eine universelle \cov\ von $X$, falls f�r jede \cov\ $q:Z\longrightarrow X$, wobei $Z$ eine \zusman\ ist, und jedes Paar $y\in Y$, $z\in Z$ mit $p(y)=q(z)$ genau eine \cov\ $f:Y\longrightarrow Z$ \ex, so da� $f(y)=z$ und $q\circ f=p$ gilt, d.h. folgendes Diagramm kommutiert:
$$
\xymatrix
{
Y\ar[dr]_p\ar@{.>}[rr]^f&&Z\ar[ld]^q\\
&X
}
$$
\end{df}
Aus der universellen Eigenschaft folgt, da� eine universelle \cov\ bis auf Hom�omorphie eindeutig ist.

%Q Fo S. 32 Thm 5.2, 5.3
Zu jeder \zus en \mannig\ \ex\ eine \zus e, einfach \zusman\ $\widetilde X$ mit einer \cov\ $p:\widetilde X\longrightarrow X$. 
Nun ist jede einfach \zus e \cov\ eine universelle \cov, also ist $(\widetilde X,p)$ eine universelle \cov\ von $X$.\footnote{siehe Theorem 5.2, 5.3 auf S.32 in \cite{Forster}}
\begin{df}
Seien $X$ und $Y$ \zusman en, und sei $p:Y\longrightarrow X$ eine \cov. Eine \textbf{Decktransformation} der \cov\ $p$ ist ein \homeo smus $f:Y\longrightarrow Y$, f�r den $p=p\circ f$ gilt.
\[\xymatrix{
Y\ar[rr]^f\ar[rd]_p&&Y\ar[ld]^p\\
&X
}\]
Die Decktransformationen von $p$ bilden eine Gruppe, diese bezeichnen wir mit $\Aut(p)$.
\end{df}
%Q Fo S.34 Thm 5.6
\begin{thm}
\label{covdeckfund}
Sei $(\widetilde X,p)$ eine universelle \cov\ von $X$, dann ist $\Aut(p)$ isomorph zur Fundamentalgruppe $\pi_1(X)$ von $X$.
\end{thm}
Beweis: Theorem 5.6 auf S.34 in \cite{Forster}\bigskip\\
Sei $X$ eine \zusman\ und $(\widetilde X,p)$ eine universelle \cov\ von $X$. Sei $q:Y\longrightarrow X$ eine weitere \cov, wobei auch $Y$ eine \zusman\ sei. Dann \ex\ eine \cov\ $f:\widetilde X\longrightarrow Y$ mit $p=q\circ f$.
\[\xymatrix{
\widetilde X\ar[rr]^f\ar[rd]_p&&Y\ar[ld]^q\\
&X
}\]
$\widetilde X$ ist eine einfach \zus e \cov\ von $Y$, also eine universelle \cov.
$\Aut(f)$ ist eine Untergruppe von $\Aut(p)$, denn f�r $\varphi\in\Aut(f)$ gilt per Definition $f=f\circ\varphi$. Wegen $p=q\circ f$ erhalten wir
\[
p=q\circ f=q\circ f\circ\varphi=p\circ\varphi
\]
\sieg
Sei $(\widetilde X,p)$ eine universelle \cov\ von $X$, dann \ex\ zu $x,y\in\widetilde X$ mit $p(x)=p(y)$ ein $f\in\Aut(p)$ mit $f(x)=y$, d.h. $\Aut(p)$ operiert transitiv auf den Fasern von $p$. Dies sieht man leicht, wenn man die universelle Eigenschaft von $p$ gem�� Definition \ref{defunivcov} ausnutzt, indem man $Y:=\widetilde X$ und $Z:=\widetilde X$ setzt. Wir nennen eine \cov\ mit dieser Eigenschaft \textbf{galoissch}.

Wir f�hren auf $\widetilde X$ eine �quivalenzrelation ein, indem wir zwei Punkte \mbox{$x,y\in Y$} identifizieren, falls ein $\varphi\in\Aut(p)$ mit $\varphi(x)=y$ \ex. Den Quotientenraum bezeichnen wir mit $\widetilde X/\Aut(p)$. Nun gilt
\[p(x)=p(y)\Longleftrightarrow\exists\varphi\in\Aut(p)\colon\varphi(x)=y\]
Also induziert $p$ eine bijektive Abbildung
\[\widetilde p:\widetilde X/\Aut(p)\longrightarrow X\]
\begin{bem}
\label{univueberfdas}
$\widetilde p$ ist ein \homeo smus.
\end{bem}
\bew
$$
\xymatrix
{
\widetilde X/\Aut(p)\ar[rr]^{\widetilde p}&&X\\
Y\ar[u]^q\ar[urr]_p
}
$$
Die Stetigkeit von $\widetilde p$ folgt aus der Finalit�t der Quotientenabbildung $q$. Es bleibt zu zeigen, da� $\widetilde p^{-1}$ stetig ist. Dazu zeigen wir, da� f�r jedes $\overline x\in\widetilde X/\Aut(p)$ das Bild einer Umgebung von $\overline x$ unter $\widetilde p$ eine Umgebung von $\widetilde p(x)$ ist. Sei nun $\overline x\in\widetilde X/\Aut(p)$, dann hat $p(x)$ eine Umgebung $U$, so da� $p^{-1}(U)=\bigcup_{j\in J}V_j$ disjunkte Vereinigung offener Mengen ist und die $p_{|V_j}$ \homeo smen sind. Nun liegt $x$ in einem dieser $V_j$, daher fixieren wir von nun an dieses $V_j$.

Sei $O$ eine Umgebung von $\overline x$ in $\widetilde X/\Aut(p)$, dann ist $q^{-1}(O)$ eine Umgebung von $x$, und $W:=q^{-1}(O)\cap V_j$ ist ebenfalls eine Umgebung von $x$. $p(W)$ ist eine Umgebung von $p(x)$, da $p_{|V_j}$ ein \homeo smus ist und ist enthalten in $\widetilde p(O)$, damit ist $\widetilde p(O)$ eine Umgebung von $\widetilde p(\overline x)$.\sieg
Wir wissen nun:
\begin{minilemma}
Sei $(\widetilde X,p)$ eine universelle \cov\ einer \zus en \mannig\ $X$, dann gilt $\widetilde X/\Aut(p)\cong X$.
\end{minilemma}
\begin{df}
Seien $p_1:Y_1\longrightarrow X$ und $p_2:Y_1\longrightarrow X$ \cov en. $(Y_1,p_1)$ und $(Y_2,p_2)$ hei�en hom�omorph als \cov en von $X$, falls ein \homeo smus $f:Y_1\longrightarrow Y_2$ \ex, so da� $p_2\circ f=p_1$ gilt.
\[\xymatrix{
Y_1\ar[rd]_{p_1}\ar[rr]^f&&Y_2\ar[ld]^{p_2}\\
&X
}\]
Einfacher gesagt: Jedes $y_1\in Y_1$, das �ber einem $x\in X$ liegt, wird unter $f$ auf ein $y_2\in Y_2$ abgebildet, das �ber $x$ liegt.
\end{df}
\begin{bem}
\label{eqrelisossdfjko}
Die soeben definierte Relation zwischen \cov en ist eine �quivalenzrelation.
\end{bem}
Denn: F�r die Reflexivit�t setzt man $f:=\id$, Transitivit�t erh�lt man durch komponieren. F�r die Symmetrie formt man wie folgt um:
\[p_2\circ f=p_1=p_1\circ f^{-1}\circ f\Longrightarrow p_2=p_1\circ f^{-1}\]
\sieg
Sei $(\widetilde X,p)$ eine universelle \cov\ von $X$, und sei $q:Y\longrightarrow X$ eine weitere \cov. Dann \ex\ eine \cov\ $f:\widetilde X\longrightarrow Y$ mit $p=q\circ f$.

Seien $\overline x,\overline y\in\widetilde X/\Aut(f)$ mit $\overline x=\overline y$, dann gilt $f(x)=f(y)$ und wegen $p=q\circ f$ gilt auch $p(x)=p(y)$. Wir k�nnen folgende Abbildungen in naheliegender Weise repr�sentantenweise definieren:
\begin{eqnarray*}
\widetilde p:\widetilde X/\Aut(f)&\longrightarrow&X\\
\widetilde f:\widetilde X/\Aut(f)&\longrightarrow&Y
\end{eqnarray*}
Wegen Bemerkung \ref{univueberfdas} ist $\widetilde f:\widetilde X/\Aut(f)\longrightarrow Y$ ein \homeo smus, also ist $\widetilde p=q\circ\widetilde f$ eine \cov, die hom�omorph zu $q$ ist.

Unser Resultat ist nun, da� jede \cov\ von $X$ hom�omorph zu $\widetilde X/H$ ist, wobei $H$ eine Untergruppe von $\Aut(p)$ ist.
\begin{satz}
\label{covquotientenmaster}
Sei $p:\widetilde X\longrightarrow X$ eine universelle \cov\ von $X$, und sei $q:Y\longrightarrow X$ eine weitere \cov\ vom Grad $n$. Dann \ex\ eine \cov\ $f:\widetilde X\longrightarrow Y$ mit $p=q\circ f$. $\Aut(f)$ hat dann endlichen Index in $\Aut(p)$, und es gilt $[\Aut(p):\Aut(f)]=n$.
\end{satz}
\[\xymatrix{
\widetilde X\ar[rr]^f\ar[rd]_p&&Y\ar[ld]^q\\
&X
}\]
\bew
Sei $M$ die Menge der \cov en $\varphi:\widetilde X\longrightarrow Y$, f�r die $p=q\circ\varphi$ gilt. $\Aut(p)$ operiert auf $M$ via
\begin{eqnarray*}
\Aut(p)\times M&\longrightarrow&M\\
(\tau,\varphi)&\longmapsto&\varphi\circ\tau
\end{eqnarray*}
Es gilt $f\in M$, der Stabilisator von $f$ ist per Definition $\Aut(f)$. Die Operation ist transitiv, denn seien $\varphi_1,\varphi_2\in M$, und sei $y\in Y$. W�hle $x_1\in\varphi_1^{-1}(y)$ und $x_2\in\varphi_2^{-1}(y)$, dann \ex\ ein $\tau\in\Aut(p)$ mit $\tau(x_1)=x_2$, da $p$ galoissch ist. Es gilt $\varphi_2\circ\tau(x_1)=\varphi_1(x_1)$, daraus folgt $\varphi_2\circ\tau=\varphi_1$.

Sei $x\in\widetilde X$, dann \ex\ wegen der Universalit�t von $p$ zu jedem \mbox{$y\in q^{-1}(p(x))$} genau eine \cov\ $\varphi:\widetilde X\longrightarrow Y$ mit $\varphi(x)=y$ und $p=q\circ\varphi$. Die Menge $q^{-1}(p(x))$ hat nach Voraussetzung genau $n$ Elemente, damit hat auch $M$ genau $n$ Elemente.

Nun operiert $\Aut(p)$ transitiv auf $M$, also hat der Orbit von $f$ genau $n$ Elemente. Nach Bemerkung \ref{stabilisatormaster} gilt $[\Aut(p):\Aut(f)]=n$.\sieg
\subsection{\rf n}
%Q Fo S.29 4.23, S.30 4.25 - um zu sehen, da� t eine � ist.
Seien $X$ und $Y$ kompakte \rf n, und sei $f:X\longrightarrow Y$ eine nicht-konstante \hol e Abbildung. Sei $A$ die Menge der kritischen Punkte von $f$, dann ist $A$ endlich und
\[
f_{|A}:X\setminus A\longrightarrow Y\setminus f(A)
\]
ist eine \cov.\footnote{siehe S.29 in \cite{Forster}} Wir nennen $f$ eine \textbf{verzweigte \hol e \cov}.\bigskip\\
Nun kommt eine fundamentale Tatsache aus der Theorie der \rf n, die wir sp�ter ben�tigen werden und deren Beweis nicht trivial ist:
%Q Foster S.52
\begin{thm}
\label{forstercovermaster}
Seien $X$, $Y$ und $Z$ kompakte \rf n und seien $\pi:Y\longrightarrow X$, $\tau:Z\longrightarrow X$ verzweigte \hol e \cov en. Sei $S\subset X$ eine endliche Teilmenge. Dann kann jede faser-erhaltende bi\hol e Abbildung
\[\varphi:Y\setminus\pi^{-1}(S)\longrightarrow Z\setminus\tau^{-1}(S)\]
zu einer faser-erhaltenden bi\hol en Abbildung $\widetilde\varphi:Y\longrightarrow Z$, d.h. $\tau\circ\widetilde\varphi=\pi$, fortgesetzt werden.
\end{thm}
Beweis: Theorem 8.5 auf S.52 in \cite{Forster}\bigskip\\
Sei $X$ eine vollst�ndige Kurve �ber $\C$, dann k�nnen wir diese als kompakte \rf\ auffassen, und jeder Morphismus zwischen vollst�ndigen Kurven gibt eine \hol e Abbildung. Umgekehrt l��t sich jede kompakte \rf\ als vollst�ndige Kurve realisieren. Jede \hol e Abbildung zwischen kompakten \rf n ist algebraisch, d.h. ein Morphismus zwischen vollst�ndigen Kurven.

In naheliegender Weise k�nnen wir einer verzweigten \cov\ einen \textbf{Grad} zuordnen. Der K�rper der meromorphen Funktionen ${\cal M}(X)$ ist isomorph zum K�rper der rationalen Funktionen $K(X)$. Aus dem folgenden Theorem folgt damit, da� uns ein endlicher Morphismus vom Grad $d$ eine verzweigte \hol e \cov\ vom Grad $d$ gibt.
\begin{thm}
\label{alg_fkt_grad}
Sei $\pi:Y\longrightarrow X$ eine \hol e \cov\ vom Grad $d$, dann induziert diese einen \khom
\begin{eqnarray*}
\pi^*:{\cal M}(X)&\longrightarrow&{\cal M}(Y)\\
f&\longmapsto&f\circ\pi
\end{eqnarray*}
Die \KE\ ${\cal M}(Y):{\cal M}(X)$ ist eine algebraische Erweiterung vom Grad $d$.
\end{thm}
Beweis: Theorem 8.3 auf S.50 in \cite{Forster}

\section{Modulk�rper}
\label{abschnitt_modulk}
In diesem Abschnitt wird der Modulk�rper eines endlichen Morphismus \\$t:X\longrightarrow\Pc$ definiert. Wir interessieren uns haupts�chlich f�r den Fall \mbox{$C=\C$}. Das Hauptresultat dieses Abschnitts wird sein, da� der Modulk�rper eines Morphismus $t:X\longrightarrow\P1$ ein Zahlk�rper ist, falls dessen kritische Werte in $\dreier$ liegen.

Zur Definition des Modulk�rpers erinnere ich an die Definition \ref{modulk_dep}.
\begin{df}
Sei $X$ eine \var\ �ber $C$. Setze
\[
U(X):=\{\sigma\in\Aut(C)\mid X^\sigma\cong X\}
\]
Der Fixk�rper $M(X):=C^{U(X)}$ hei�t \textbf{absoluter Modulk�rper} von $X$.
\end{df}
\subsection{Modulk�rper eines endlichen Morphismus}
\begin{df}
\label{moduli_covering}
Sei $t:X\longrightarrow\Pc$ ein endlicher Morphismus zwischen Kurven �ber $C$.
$U(X,t)\subset\Aut(C)$ besteht aus denjenigen K�rperautomorphismen $\sigma$, f�r die ein Isomorphismus $f_\sigma$ \ex, so da� folgendes Diagramm kommutiert.
\[\xymatrix{
X^\sigma\ar[rr]^{f_\sigma}\ar[d]_{t^\sigma}&&X\ar[d]^t\\
(\Pc)^\sigma\ar[rr]_{\proj(\sigma)}&&\Pc
}\]
Der Fixk�rper $M(X,t):=C^{U(X,t)}$ hei�t \textbf{Modulk�rper} von $(X,t)$.
\end{df}
Insbesondere gilt $U(X,t)\subset U(X)$.% $f_\sigma$ aus obiger Definition induziert einen K�rperautomorphismus $f_\sigma^*:K(X)\longrightarrow K(X)$. Wir erhalten folgendes kommutative Diagramm:
%\[\xymatrix{
%K(X)&&K(X)\ar[ll]_{f_\sigma^*}\\
%C(T)\ar[u]^{t^*}&&C(T)\ar[ll]^\sigma\ar[u]_{t^*}
%}\]
%Insbesondere ist $t^*(T)$ invariant unter $f_\sigma^*$.
\subsection{Modulk�rper im Komplexen}
Sei $X$ eine vollst�ndige Kurve und $f:X\longrightarrow\P1$ ein endlicher Morphismus. $f$ gibt uns eine verzweigte \hol e \cov. Sei $S\subset\P1$ die Menge der kritischen Werte von $f$, dann ist
\[t_S:X\setminus t^{-1}(S)\longrightarrow\P1\setminus S\]
wobei $t_S$ die naheliegende Einschr�nkung von $t$ sei, eine \hol e \cov. Der Grad der \cov\ $t_S$ stimmt mit dem Grad des endlichen Morphismus $t$ �berein.
%Q K�ck Prop 3.1
%Q FO S.23 - f�r biholomorph
\begin{lem}
\label{isoklasseninj}
Seien $t_1:X_1\longrightarrow\P1$ und $t_2:X_2\longrightarrow\P1$ endliche Morphismen zwischen vollst�ndigen Kurven, und sei $S\subset\P1$ eine endliche Teilmenge, so da� die kritischen Werte von $t_1$ und $t_2$ in $S$ enthalten sind. Dann sind $t_1$ und $t_2$ genau dann isomorph, wenn die zugeordneten \hol en \cov en $t_{1S}$ und $t_{2S}$ hom�omorph sind.
\end{lem}
\bew
Der �bergang vom algebraischen zum analytischen ist funktoriell, also folgt aus $t_1$ isomorph zu $t_2$, da� $t_{1S}$ und $t_{2S}$ hom�omorph sind.

Sei nun umgekehrt $t_{1S}$ hom�omorph zu $t_{2S}$, d.h. es \ex\ ein \homeo smus $f:X_1\setminus t_1^{-1}(S)\longrightarrow X_2\setminus t_2^{-1}(S)$ mit $t_{2S}\circ f=t_{1S}$. Nun sind $t_{1S}$ und $t_{2S}$ als \cov en lokal bi\hol e Abbildungen. Daraus folgt, da� auch $f$ lokal bi\hol\ ist, denn sei $x\in X_1\setminus t_1^{-1}(S)$, dann hat $y:=t_{1S}(x)\in\P1\setminus S$ gem�� Definition \ref{defcov} Umgebungen $U_1$, $U_2$, so da� $x$ eine Umgebung $V_1$ und $f(x)$ eine Umgebung $V_2$ hat mit $t_{1S|V_1}:V_1\longrightarrow U_1$ und $t_{2S|V_2}:V_2\longrightarrow U_2$ bi\hol. Wir k�nnen annehmen, da� $U:=U_1=U_2$ gilt, da wir ansonsten kleinere Umgebungen w�hlen k�nnen. 
\[\xymatrix{
V_1\ar[rr]^{f_{|V_1}}\ar[rd]_{t_{1S|V_1}}&&V2\ar[ld]^{t_{2S|V_2}}\\
&U
}\]
%Q GRIFFITHS
Nun gilt $t_{2S|V_2}\circ f_{|V_1}=t_{1S|V_1}$, daraus folgt $f_{|V_1}=t_{1S|V_1}\circ t_{2S|V_2}^{-1}$. $f$ ist lokal bi\hol\ und ein \homeo smus, also ist $f$ bi\hol. Nach Theorem \ref{forstercovermaster} \ex\ eine Fortsetzung $\widetilde f:X_1\longrightarrow X_2$ von $f$, die faser-erhaltend, d.h. \mbox{$t_2\circ\widetilde f=t_1$}, und bi\hol\ ist. Jede \hol e Abbildung zwischen vollst�ndigen Kurven ist algebraisch, also ist $\widetilde f$ der gesuchte Isomorphismus.\sieg
\begin{lem}
\label{endlvieleisos}
Sei $S\subset\P1$ eine endliche Teilmenge von abgeschlossenen Punkten und sei $d\geq 1$ eine nat�rliche Zahl. Dann existieren nur endlich viele Isomorphieklassen von Paaren $(X,t)$, wobei $X$ eine vollst�ndige Kurve und $t:X\longrightarrow\P1$ ein endlicher Morphismus vom Grad $d$, dessen kritische Werte in $S$ liegen, ist.
\end{lem}
\bew
Sei $t:X\longrightarrow\P1$ ein endlicher Morphismus, dessen kritische Werte in $S$ liegen, dann erhalten wir eine \hol e \cov\ \mbox{$t_S:X\setminus t^{-1}(S)\longrightarrow\P1\setminus S$}. Wir ordnen nun jeder Isomorphieklasse $(X,t)$ die \homeo eklasse \\\mbox{$(X\setminus t^{-1}(S),t_S)$} zu. Diese Zuordnung ist nach Lemma \ref{isoklasseninj} injektiv.

Nun ist zu zeigen, da� es nur endlich viele \homeo eklassen von \cov en von $\P1\setminus S$ vom Grad $d$ gibt. Sei $p:\widetilde S\longrightarrow\P1\setminus S$ eine universelle \cov\ von $\P1\setminus S$, und sei $t:Y\longrightarrow\P1\setminus S$ eine \cov\ von $\P1\setminus S$, dann ist die \cov\ $\widetilde S/\Aut(t)\longrightarrow\P1\setminus S$ isomorph zu $t:Y\longrightarrow\P1\setminus S$. Nach Satz \ref{covquotientenmaster} stimmt der Grad der \cov\ $t$ mit dem Index von $\Aut(t)$ in $\Aut(p)$ �berein. Wir interessieren uns nun f�r \cov en vom Grad $d$, von denen es h�chstens so viele \homeo eklassen gibt, wie $\Aut(p)$ Untergruppen vom Index $d$ hat.

%Nun ist zu zeigen, da� $\Aut(p)$ nur endlich viele Untergruppen vom Index $d$ hat.
Nach Theorem \ref{covdeckfund} ist die Fundamentalgruppe $\pi_1(\P1\setminus S)$ von $\P1\setminus S$ isomorph zu $\Aut(p)$. $\pi_1(\P1\setminus S)$ ist eine endlich erzeugte freie Gruppe\footnote{siehe Aufgabe 5.7.A2 in \cite{Zieschang}} vom Rang $\#S-1$, nach Satz \ref{hall_freie_gruppe} hat diese nur endlich viele Untergruppen vom Index $d$.\sieg
%Q K�ck Cor 3.2
\begin{lem}
\label{koeck32}
Sei $X$ eine vollst�ndige Kurve �ber $\C$, $t:X\longrightarrow\P1$ ein endlicher Morphismus und $K$ ein Unterk�rper von $\C$, so da� alle kritischen Werte von $t$ $K$-rational sind. Dann ist die \KE\ $M(X,t):K$ endlich algebraisch.
\end{lem}
\bew
Sei $\sigma\in\Aut(\C:K)$, dann erhalten wir einen endlichen Morphismus
\[
t(\sigma):=\proj(\sigma)\circ t^\sigma
\]
d.h. das folgende Diagramm kommutiert:
\[\xymatrix{
X^\sigma\ar[d]_{t^\sigma}\ar[rrd]^{t(\sigma)}\\
{\P1}^\sigma\ar[rr]_{\proj(\sigma)}&&\P1
}\]
$\proj(\sigma)$ hat nach Bemerkung \ref{isokeinkrit} keine kritischen Werte und ist ein endlicher Morphismus vom Grad $1$. Nach Bemerkung \ref{qratinv} sind $K$-rationale Punkte invariant unter $\proj(\sigma)$, also sind die kritischen Werte von $t(\sigma)$ die gleichen wie die von $t$. Sei $d$ der Grad von $t$, dann hat $t(\sigma)$ ebenfalls den Grad $d$. Wir erhalten also eine Gruppenoperation von $\Aut(\C:K)$ auf den Isomorphieklassen.

$t$ hat nach Folgerung \ref{endlkritendl} nur endlich viele kritische Punkte.
Sei $S$ die Menge der kritischen Werte von $t$, dann existieren nach Lemma \ref{endlvieleisos} nur endlich viele Isomorphieklassen von endlichen Morphismen nach $\P1$ vom Grad $d$, deren kritische Werte in $S$ liegen. Insbesondere ist der Orbit der Isomorphieklasse $(X,t)$ unter der Operation von $\Aut(\C:K)$ endlich. Der Stabilisator von $(X,t)$ hat nach Bemerkung \ref{stabilisatormaster} endlichen Index in $\Aut(\C:K)$. Nun besteht der Stabilisator von $(X,t)$ aus denjenigen $\sigma\in\Aut(\C:K)$, f�r die ein Isomorphismus $f_\sigma$ \ex, so da� folgendes Diagramm kommutiert:
\[\xymatrix{
X^\sigma\ar[rr]^{f_\sigma}\ar[rrd]^{t(\sigma)}&&X\ar[d]^t\\
&&\P1
}\]
Wie man nun leicht sieht, gilt $\stab(X,t)\subset U(X,t)$.\footnote{siehe Definition \ref{moduli_covering}} Insbesondere hat $U(X,t)$ damit endlichen Index in $\Aut(\C:K)$. Nach Satz \ref{lemma_vrf} gilt $\C^{\Aut(\C/K)}=K$, und nach Lemma \ref{indendldegendl} ist die \KE\ $\C^{U(X,t)}:\C^{\Aut(\C/K)}$ endlich algebraisch.\sieg
Nun kommen wir zu unserem Hauptresultat, das ein Spezialfall von Lemma \ref{koeck32} ist: Inbesondere ist $M(X,t)$ eine endliche Erweiterung von $\Q$, also ein Zahlk�rper, falls die kritischen Werte von $t$ in $\dreier$ liegen.

Als n�chstes suchen wir eine Absch�tzung f�r den Grad der \KE\ $M(X,t):\Q$. Sei $G$ eine freie Gruppe vom Rang $2$, dann berechnet sich die Anzahl $M_d$ der Untergruppen vom Index $d$ nach Satz \ref{formelhall} mit der folgenden Rekursionsformel:
\begin{eqnarray*}
M_1&=&1\\
M_d&=&d(d!)-\sum\limits_{i=1}^{d-1}(d-i)!M_i
\end{eqnarray*}
\begin{bem}
\label{gradabsch1}
Sei $X$ eine vollst�ndige Kurve �ber $\C$, $t:X\longrightarrow\P1$ ein endlicher Morphismus vom Grad $d$, dessen kritische Werte in $\dreier$ liegen. Dann gilt
\[
[M(X,t):\Q]\leq M_d
\]
\end{bem}
Der Beweis greift auf Einzelheiten der Beweise von Lemma \ref{endlvieleisos} und Lemma \ref{koeck32} zur�ck.\smallskip\\
\bew
$\pi_1(\P1\setminus\dreier)$ ist eine freie Gruppe vom Rang $2$. Es gibt h�chstens so viele \homeo eklassen von \cov en von $\P1\setminus\dreier$ vom Grad $d$, wie $\pi_1(\P1\setminus\dreier)$ Untergruppen vom Index $d$ hat. Insbesondere hat der Orbit von $(X,t)$ unter der Operation von $\Aut(\C:\Q)$ h�chstens $M_d$ Elemente. Es gilt $\stab(X,t)=U(X,t)$, wie man leicht sieht, damit ist der Index von $U(X,t)$ in $\Aut(\C:\Q)$ nach Bemerkung \ref{stabilisatormaster} h�chstens $M_d$. Nach Lemma \ref{indendldegendl} gilt
\[[M(X,t):\Q]\leq[\Aut(\C:\Q):U(X,t)]\leq M_d\]

\sieg

\def\Z{\mathbbm{Z}}
\def\ord{\mathop{\rm ord}\nolimits}
\def\Div{\mathop{\rm Div}\nolimits}
\section{Vorbereitungen III}
\subsection{Verzweigung}
\label{kapitel_verzw}
Sei $X$ eine Kurve �ber $C$, dann ist f�r jeden abgeschlossenen Punkt $x\in X$ der lokale Ring $\O_{X,x}$ ein diskreter Bewertungsring. Sei $K(X)$ der K�rper der rationalen Funktionen auf $X$, dann gilt $K(X)\cong Q(\O_{X,x})$. Im Folgenden fasse ich $\O_{X,x}$ als Unterring von $K(X)$ auf. Es \ex\ eine Bewertung $v_x$ auf $K(X)$ mit $v_x(c)=0$ f�r alle $c\in C$, so da�
\[\O_{X,x}=\{f\in K(X)\mid v_x(f)\geq 0\}\cup\{0\}\]
gilt. Sei $\m_x$ das maximale Ideal von $\O_{X,x}$, dann gilt
\[\m_x=\{f\in K(X)\mid v_x(f)>0\}\cup\{0\}\]
Sei $t\in\m_x\setminus\m_x^2$, dann wird $\m_x$ nach Lemma \ref{diff_diskret_modul} von $t$ erzeugt. F�r $0\not=f\in\m_x$ \ex\ ein $g\in\O_{X,x}$, so da� $f=g\cdot t$ gilt. Per Definition einer diskreten Bewertung gilt $v_x(f)=v_x(g)+v_x(t)$. Da das f�r alle $0\not=f\in\m_x$ gilt, mu� $v_x(t)=1$ gelten. Ein solches $t$ hei�t \textbf{lokaler Parameter} in $x$.

$\m_x^n$ wird von $t^n$ erzeugt, f�r $f\in\O_{X,x}\setminus\{0\}$ gilt $v_x(f)=\max\{n\in\nat\mid f\in\m_x^n\}$. Wir nennen $v_x(f)$ die \textbf{Nullstellenordnung} von $f$ in $x$. Sei nun $f\in K(X)\setminus\{0\}$, dann existieren $a,b\in\O_{X,x}\setminus\{0\}$, so da� $f=\frac ab$ gilt. Es gilt
\[v_x(f)=v_x(a)-v_x(b)\]
Wir nennen $\ord_x(f):=v_x(f)$ die \textbf{Ordnung} von $f$ in $x$ und sagen, da� $f$ einen \textbf{Pol} in $x$ hat, falls $\ord_x(f)<0$ gilt, in diesem Fall ist $-\ord_x(f)$ die \textbf{Polordnung}.\bigskip
%Q Ha S.299
\begin{df}
Seien $X$ und $Y$ vollst�ndige Kurven �ber $C$, und sei \mbox{$f:X\longrightarrow Y$} ein endlicher Morphismus. Sei $P\in X$ ein abgeschlossener Punkt, dann induziert $f$ einen lokalen $C$-Algebrenhomomorphismus
\[\varphi:\O_{Y,f(P)}\longrightarrow\O_{X,P}\]
Sei $t\in\O_{Y,f(P)}$ ein lokaler Parameter in $f(P)$. $\O_{X,P}$ ist ein diskreter Bewertungsring, sei $v_P$ die zugeh�rige Bewertung, dann hei�t
\[e_P:=v_P(\varphi(t))\]
der \textbf{Verzweigungsindex} von $f$ in $P$. $f$ hei�t verzweigt in $P$, falls $e_P>1$ gilt, und unverzweigt in $P$, falls $e_P=1$ gilt.
\end{df}
$e_P$ ist wohldefiniert, da $\varphi$ injektiv ist, also insbesondere $\varphi(t)\not=0$ gilt.
\begin{bem}
\label{verzequivkrit}
$f$ ist genau dann in $P$ verzweigt, wenn $P$ ein kritischer Punkt von $f$ ist.
\end{bem}
\bew
Sei $\varphi:\O_{Y,f(P)}\longrightarrow\O_{X,P}$ der durch $f$ induzierte lokale \rhom, und sei $t\in\O_{Y,f(P)}$ ein lokaler Parameter $f(P)$. \zz $e_P>1$ gilt genau dann, wenn die Cotangentialabbildung $\widetilde\varphi:\m_{f(P)}/\m_{f(P)}^2\longrightarrow\m_P/\m_P^2$ die Nullabbildung ist. (siehe Definition \ref{kritpktdef})

Nun ist $e_P>1$ �quivalent zu $\varphi(t)\in\m_{P}^2$. Das wiederum ist �quivalent dazu, da� $\widetilde\varphi$ die Nullabbildung ist.\sieg
\begin{bem}
\label{verzwmult}
Seien $X$, $Y$ und $Z$ Kurven �ber $C$ und seien $f:X\longrightarrow Y$ und $g:Y\longrightarrow Z$ endliche Morphismen von $C$-\var en. Ist $P\in X$ ein abgeschlossener Punkt, dann ist der Verzweigungsindex von $g\circ f$ in $P$ das Produkt der Verzweigungsindizes von $f$ in $P$ und von $g$ in $f(P)$.
\end{bem}
Dies sieht man leicht, wenn man die lokalen Ringhomomorphismen komponiert.\sieg
Der folgende Satz ist ein Resultat, das wir bereits aus der Theorie der Riemannschen Fl�chen  kennen:
\begin{minilemma}
Eine nicht-konstante meromorphe Funktion auf einer kompakten Riemannschen Fl�che nimmt jeden Wert mit Vielfachheiten gleich oft an.
\end{minilemma}
\begin{satz}
\label{alles_gleichoft}
Sei $f:X\longrightarrow Y$ ein endlicher Morphismus zwischen Kurven �ber $C$ vom Grad $n$. Sei $Q\in Y$ ein abgeschlossener Punkt, dann gilt
\[
\sum_{P\in f^{-1}(Q)}e_P=n
\]
$e_P$ bezeichnet den Verzweigungsindex von $f$ im Punkt $P$.
\end{satz}
Ein endlicher Morphismus vom Grad $n$ nimmt also jeden Wert mit Vielfachheiten genau $n$-mal an. Falls $Q$ kein kritischer Wert von $f$ ist, hat $Q$ genau $n$ Urbilder (siehe Bemerkung \ref{verzequivkrit}). Wir folgern weiterhin, da� der Grad eines endlichen Morphismus eine obere Schranke f�r den Verzweigungsindex in einem Punkt ist, d.h. $e_P\leq n$.
\begin{df}
Sei $f:X\longrightarrow Y$ ein endlicher Morphismus zwischen Kurven �ber $C$ vom Grad $n$. Sei $P\in X$ ein abgeschlossener Punkt, dann hei�t $f$ in $P$ \textbf{total verzweigt}, falls $e_P=n$ gilt.
\end{df}
\subsection{Satz von Riemann-Roch}
Das Ziel von diesem Abschnitt ist, zu einer gegebenen vollst�ndigen Kurve $X$ und einem abgeschlossenen Punkt $P$ eine nicht-konstante rationale Funktion zu finden, die ihren einzigen Pol im Punkt $P$ hat.
\begin{df}
Sei $X$ eine Kurve �ber $C$ und $\widetilde X\subset X$ die Menge aller abgeschlossenen Punkte von $X$. Sei
\[
\Div(X):=\bigoplus\limits_{x\in\widetilde X}\Z
\]
die von den Elementen aus $\widetilde X$ erzeugte freie abelsche Gruppe. Ein \textbf{Divisor} auf $X$ ist dann ein Element aus $\Div(X)$.
\end{df}
Ein Divisor $D$ auf $X$ ist also eine formale endliche Linearkombination von abgeschlossen Punkten, d.h.
\[
D=\sum\limits_{x\in\widetilde X}n_x\cdot x
\]
mit $n_x\in\Z$ und $n_x\not=0$ f�r endlich viele $x\in\widetilde X$. Seien $D=\sum n_x\cdot x$ und $D'=\sum n_x'\cdot x$ zwei Divisoren auf $X$. Wir setzen
\[
D\leq D':\Longleftrightarrow n_x\leq n_x'\textrm{ f�r alle }x\in\widetilde X
\]
Wir erhalten damit eine partielle Ordnungrelation auf $\Div(X)$.
\bigskip\\
%Q Ha S.131
Sei $f\in K(X)$ eine rationale Funktion, dann gilt $\ord_x(f)\not=0$ f�r endlich viele $x\in\widetilde X$. Wir erhalten also einen Divisor $(f)$ auf $X$:
\[
(f):=\sum\limits_{x\in\widetilde X}\ord_x(f)\cdot x
\]
Ein Divisor der Form $(f)$ hei�t \textbf{Hauptdivisor}.
\begin{df}
Sei $D$ ein Divisor auf $X$. Dann bezeichnet
\[
\deg(D):=\sum\limits_{x\in\widetilde X}n_x
\]
den \textbf{Grad} von $D$.
\end{df}
%Q Ha S.132
\begin{bem}
Sei $X$ eine vollst�ndige Kurve, und sei $D$ ein Hauptdivisor auf $X$, dann gilt $\deg D=0$.
\end{bem}
%Q L�t S.111
\begin{df}
Sei $X$ eine vollst�ndige Kurve, dann ist der $C$-Vektorraum $\Omega_{X/\Spec C}$ endlich-dimensional. Die nat�rliche Zahl
\[
g:=\dim_C\Omega_{X/\Spec C}
\]
hei�t das \textbf{Geschlecht} von $X$.
\end{df}
Der folgende Satz ist eine Absch�tzung, die direkt aus dem Satz von Riemann-Roch folgt.
\begin{satz}
\label{satz_rr}
Sei $X$ eine vollst�ndige Kurve vom Geschlecht $g$, und sei $D$ ein Divisor auf $X$. Setze
\[
L(D):=\{f\in K(X)\mid(f)+D\geq0\}
\]
dann ist $L(D)$ ein endlich-dimensionaler $C$-Vektorraum und es gilt
\[
\dim_C L(D)\geq\deg(D)+1-g
\]
\end{satz}
\sieg
Nun k�nnen wir unser Hauptresultat formulieren:
\begin{fol}
\label{fkt_mit_polordng}
Sei $X$ eine vollst�ndige Kurve vom Geschlecht $g$, und sei $P\in X$ ein abgeschlossener Punkt. Dann \ex\ eine nicht-konstante rationale Funktion $z\in K(X)\setminus C$, die $P$ als einzigen Pol, wobei die Polordnung in $P$ h�chstens $g+1$ ist, hat. Das hei�t
\[\ord_P(z)\geq-(g+1)\]
und f�r einen abgeschlossenen Punkt $Q\in X$ gilt
\[\ord_Q(z)\geq 0\]
\end{fol}
\bew
Setze $D:=(g+1)\cdot P$, dann gilt $\deg(D)=g+1$. Aus Satz \ref{satz_rr} folgt
\[\dim_C L(D)\geq (g+1)+1-g=2\]
Also \ex\ ein $z\in K(X)\setminus C$ mit $(f)\geq -D$.\sieg

\section{Definitionsk�rper}
\label{abschnitt_defk}
Im Folgenden ist $C$ ein \algabg er K�rper. In diesem Abschnitt werden wir sehen, da� f�r einen Morphismus $t:X\longrightarrow\Pc$ die Kurve $X$ und der Morphismus $t$ �ber einer endlichen Erweiterung des Modulk�rpers definiert sind.

F�r den Beweis des Satzes von Belyi ben�tigen wir den Spezialfall $C=\C$.
\begin{df}
\label{supermoduli}
Sei $t:X\longrightarrow\Pc$ ein endlicher Morphismus, und sei $P\in X$ ein abgeschlossener Punkt. $U(X,t,P)$ bezeichnet diejenigen K�rperautomorphismen $\sigma\in\Aut(C)$, f�r die ein Isomorphismus $f_\sigma:X\longrightarrow X$ \ex, so da� $f_\sigma(P)=P$ gilt und folgendes Diagramm kommutiert.
\[\xymatrix{
X^\sigma\ar[rr]^{f_\sigma}\ar[d]_{t^\sigma}&&X\ar[d]^t\\
(\Pc)^\sigma\ar[rr]^{\proj(\sigma)}&&\Pc
}\]
\end{df}
Ich erinnere daran, da� die zugrundeliegenden Schemata von $X$ und $X^\sigma$ identisch sind. Auch die Morphismen $t$ und $t^\sigma$ sind als Morphismen zwischen Schemata identisch.

Insbesondere gilt $U(X,t,P)\subset U(X,t)$\footnote{siehe Definition \ref{moduli_covering}}. Den Fixk�rper von $U(X,t,P)$ bezeichnen wir mit $M(X,t,P)$.
\begin{bem}
F�r $\sigma\in U(X,t,P)$ ist $f_\sigma$ sogar eindeutig bestimmt.
\end{bem}
Denn: Sei $g_\sigma$ ein weiterer Isomorphismus mit den beiden Eigenschaften von $f_\sigma$. Dann gilt $f_\sigma\circ g_\sigma^{-1}(P)=P$ und $g_\sigma^{-1}\circ f_\sigma(P)=P$.
\[\xymatrix{
X\ar[rr]^{g_\sigma^{-1}}\ar[d]^t&&X^\sigma\ar[rr]^{f_\sigma}\ar[d]^{t^\sigma}&&X\ar[d]^t&X\ar[rr]^{f_\sigma\circ g_\sigma^{-1}}\ar[rd]_t&&X\ar[ld]^t\\
\Pc\ar[rr]^{\proj(\sigma)^{-1}}&&(\Pc)^\sigma\ar[rr]^{\proj(\sigma)}&&\Pc&&\Pc
}\]
Nach Lemma \ref{algcovgalois} gilt $f_\sigma\circ g_\sigma^{-1}=\id$ und $g_\sigma^{-1}\circ f_\sigma=\id$, daraus folgt $g_\sigma=f_\sigma$.\sieg
%Q K�ck S7
\begin{bem}
\label{op_uxtp}
Sei $\sigma\in U(X,t,P)$, dann induziert das eindeutig gegebene $f_\sigma$ einen K�rperautomorphsimus $f_\sigma^*:K(X)\longrightarrow K(X)$. Insbesondere kommutiert das folgende Diagramm:
\[\xymatrix{
K(X)&&K(X)\ar[ll]_{f_\sigma^*}\\
C(T)\ar[u]^{t^*}&&C(T)\ar[u]_{t^*}\ar[ll]^{\sigma}
}\]
Wir erhalten also eine Gruppenoperation von $U(X,t,P)$ auf $K(X)$.
\end{bem}
\begin{lem}
\label{def_endliche_erw}
Sei $t:X\longrightarrow\Pc$ ein endlicher Morphismus, und sei \mbox{$Q\in\Pc$} ein $\Q$-rationaler Punkt. Sei $P\in t^{-1}(Q)$, dann ist die \KE\\ $M(X,t,P):M(X,t)$ endlich algebraisch.
\end{lem}
\bew
Sei $\sigma\in U(X,t)$. $Q$ ist $\Q$-rational, also ist $Q$ nach Bemerkung \ref{qratinv} invariant unter $\proj(\sigma)$. Zu $\sigma$ \ex\ ein Morphismus $f_\sigma:X^\sigma\longrightarrow X$, so da� das Diagramm
\[\xymatrix{
X^\sigma\ar[rr]^{f_\sigma}\ar[d]_{t^\sigma}&&X\ar[d]^t\\
(\Pc)^\sigma\ar[rr]_{\proj(\sigma)}&&\Pc
}\]
kommutiert, damit ist $t^{-1}(Q)$ invariant unter $f_\sigma$. Ich erinnere daran, da� die Morphismen $t$ und $t^\sigma$ als Morphismen zwischen Schemata identisch sind.

Wir betrachten den Quotienten $t^{-1}(Q)/\Aut(t)$, wobei wir zwei Punkte $P_1$ und $P_2$ aus der Faser von $Q$ identifizieren, falls ein $\varphi\in\Aut(t)$ \ex, so da� $\varphi(P_1)=P_2$ gilt. $U(X,t)$ operiert auf $t^{-1}(Q)/\Aut(t)$ via
\begin{eqnarray*}
U(X,t)\times t^{-1}(Q)/\Aut(t)&\longrightarrow&t^{-1}(Q)/\Aut(t)\\
(\sigma,\overline P)&\longmapsto&\overline{f_\sigma(P)}
\end{eqnarray*}
Diese Definition ist unabh�ngig von der Wahl von $P$ und $f_\sigma$, denn sei $\overline P_1=\overline P_2$ und seien $f_\sigma$ und $g_\sigma$ gem�� der Definition von $U(X,t)$ gew�hlt. Es \ex\ ein $r\in\Aut(t)$ mit $r(P_1)=P_2$. Setze $s:=g_\sigma\circ r\circ f_\sigma^{-1}$, dann ist $s$ ein Isomorphismus. Zu zeigen ist $s\in\Aut(t)$ und $s(f_\sigma(P_1))=g_\sigma(P_2)$, damit w�re die Wohldefiniertheit gezeigt. In den folgenden Umformungen werde ich nicht zwischen $t$ und $t^\sigma$ unterscheiden.
\begin{eqnarray*}
t\circ s&=&t\circ (g_\sigma\circ r\circ f_\sigma^{-1})\\
&=&(t\circ g_\sigma)\circ r\circ f_\sigma^{-1}\\
&=&(\proj(\sigma)\circ t)\circ r\circ f_\sigma^{-1}\\
&=&\proj(\sigma)\circ (t\circ r)\circ f_\sigma^{-1}\\
&=&\proj(\sigma)\circ t\circ f_\sigma^{-1}\\
&=&t
\end{eqnarray*}
Damit ist $s\in\Aut(t)$ gezeigt. Es fehlt noch $s(f_\sigma(P_1))=g_\sigma(P_2)$.
\[s\circ f_\sigma(P_1)=(g_\sigma\circ r\circ f_\sigma^{-1})\circ f_\sigma(P_1)=g_\sigma\circ r(P_1)=g_\sigma(P_2)\]
Damit ist die Wohldefiniertheit gezeigt. Man sieht leicht, da� wir eine Gruppenoperation erhalten haben.

$U(X,t,P)$ ist der Stabilisator von $\overline P$, denn:
Jedes Element aus $U(X,t,P)$ l��t insbesondere $\overline P$ invariant. Sei nun $\sigma\in\stab(\overline P)$ dann steht $P$ in relation mit $f_\sigma(P)$, d.h. es existiert $r\in\Aut(t)$ mit $r(f_\sigma(P))=P$. $g_\sigma:=r\circ f_\sigma$ ist dann der gesuchte Isomorphismus, d.h. $g_\sigma(P)=P$. Damit gilt $U(X,t,P)=\stab(\overline P)$. Nach Bemerkung \ref{stabilisatormaster} hat $U(X,t,P)$ endlichen Index in $U(X,t)$, und nach Lemma \ref{indendldegendl} ist die \KE\ \mbox{$M(X,t,P):M(X,t)$} endlich algebraisch.\sieg
\begin{lem}
\label{laurent_zerl}
Sei $P\in X$ ein abgeschlossener Punkt und $s\in\O_{X,P}$ ein lokaler Paramter. Sei $f\in K(X)$ mit $n:=-\ord_P(f)\geq 1$, d.h. $f$ hat einen Pol der Ordnung $n$ in $P$. Dann hat $f$ eine eindeutige Darstellung
\[
f=\frac{a_{-n}}{s^n}+\frac{a_{-(n-1)}}{s^{n-1}}+\dots+a_o+r
\]
mit $a_{-n},\dots,a_0\in C$ und $r\in\m_P$.
\end{lem}
\bew
Es gilt $K(X)=Q(\O_{X,P})$, also hat $f$ eine Darstellung $f=\frac ab$ mit $a,b\in\O_{X,P}$. Wir k�nnen annehmen, da� der Bruch soweit gek�rzt ist, da� $v_p(a)=0$ und $v_P(b)=n$ gilt. Au�erdem k�nnen wir annehmen, da� der Bruch mit einer Einheit erweitert wurde, so da� $b=s^n$ gilt. Wir erhalten damit $f=\frac a{s^n}$ mit $a\in\O_{X,P}$.

Nach Lemma \ref{disk_supertrick} hat $a$ eine eindeutige Darstellung $a=a_{-n}+b$ mit $a_{-n}\in C$ und $b\in\m_P$. Wir erhalten $f=\frac{a_{-n}}{s^n}+\frac{b}{s^n}$. Wegen $v_p(b)>0$ gilt $-\ord_P(\frac{b}{s^n})<n$, also hat $f$ eine eindeutige Darstellung $f=\frac{a_{-n}}{s^n}+f'$ mit $-\ord_P(f')<n$. Diese Zerlegung f�hren wir mit $f'$ erneut durch, bis schlie�lich $v_p(f')>0$ gilt, d.h. $f'\in\m_P$.\sieg
\begin{thm}
\label{koeck22}
Sei $t:X\longrightarrow\Pc$ ein endlicher Morphismus, dann sind $X$ und $t$ �ber einer endlichen Erweiterung von $M(X,t)$ definiert.
\end{thm}
Der Funktionenk�rper von $\P1$ ist isomorph zum K�rper der rationalen Funktionen in einer Ver�nderlichen $C(T)$. Sei $K(X)$ der Funktionenk�rper von $X$, dann induziert $t:X\longrightarrow\P1$ einen \khom\ $t^*:C(T)\longrightarrow K(X)$.

Nach Folgerung \ref{endlkritendl} ist die Menge der kritischen Werte von $t$ endlich, wir w�hlen einen $\Q$-rationalen Punkt $Q\in\P1$, der kein kritischer Wert von $t$ ist. Wir w�hlen ein $P\in t^{-1}(Q)$, dann ist $P$ kein kritischer Punkt von $t$ und nach Bemerkung \ref{verzequivkrit} ist $t$ unverzweigt in $P$.

Nach Folgerung \ref{fkt_mit_polordng} \ex\ eine nicht-konstante rationale Funktion\\ $\textfrak z\in K(X)\setminus C$, so da� $P$ der einzige Pol von $\textfrak z$ ist. Wir w�hlen $\textfrak z$ so, da� die Polordnung $n:=-\ord_P(\textfrak z)$ im Punkt $P$ minimal ist.

Nach Bemerkung \ref{lok_inv_rat} k�nnen wir einen lokalen Parameter $r\in\m_Q$ so w�hlen, da� $r$ invariant unter $\Aut(C)$, also insbesondere invariant unter $U(X,t,P)$ ist. Sei $\varphi$ der durch $t$ induzierte lokale Ring\homo\ im Punkt $P$, dann ist $s:=\varphi(r)\in\m_P$ ebenfalls invariant unter $U(X,t,P)$, wie man anhand des Diagramms aus Bemerkung \ref{op_uxtp} erkennen kann. Nach Lemma \ref{laurent_zerl} hat $\textfrak z$ eine eindeutige Darstellung
\[\textfrak z=\frac{a_{-n}}{s^n}+\frac{a_{-(n-1)}}{s^{n-1}}+\dots+a_o+r\textrm{ mit } r\in\m_p\]
Setze $\textfrak z':=\frac 1{a_{-n}}\cdot\textfrak z-\frac{a_0}{a_{-n}}$. Seien $a_i'$ die Koeffizienten von $\textfrak z'$, dann gilt $a'_{-n}=1$ und $a'_0=0$.

$\textfrak z'$ ist damit eindeutig bestimmt, denn sei $\textfrak z''\in K(X)$, so da� $P$ der einzige Pol von $\textfrak z''$ ist, wobei die Polordnung minimal ist und $a''_{-n}=1$, $a''_0=0$ gilt. Dann hat $\textfrak z'-\textfrak z''$ keinen Pol $n-ter$ Ordnung in $P$ und ist wegen der Minimalit�t von $n$ damit konstant. Nun gilt $\textfrak z'-\textfrak z''\in\m_p\cap C$, also $\textfrak z'-\textfrak z''=0$. Von nun an nehmen wir an, da� $\textfrak z$ diese Eigenschaften hat, d.h. es gilt $\textfrak z=\textfrak z'$.

Der durch den \khom
\begin{eqnarray*}
z^*:C(T)&\longrightarrow&K(X)\\
T&\longmapsto&z
\end{eqnarray*}
induzierte Morphismus $z$ ist nach Satz \ref{alles_gleichoft} in $P$ total verzweigt. Setze \\\mbox{$\textfrak t:=t^*(T)\in K(X)\setminus C$}, dann gilt $C(\textfrak t,\textfrak z)\cong K(X)$. Um das zu sehen, zeigen wir, da� die \KE\ $K(X):C(\textfrak t,\textfrak z)$ Grad $1$ hat. Es \ex\ eine Kurve $Y$ �ber $C$, so da� $C(\textfrak t,\textfrak z)\cong K(Y)$ gilt. Die Inklusion $\pi^*:C(\textfrak t,\textfrak z)\hookrightarrow K(X)$ induziert einen endlichen Morphismus $\pi:X\longrightarrow Y$. Wir erhalten insgesamt folgendes kommutative Diagramm:
\[
\xymatrix{
\Pc\\
&Y\ar[lu]\ar[ld]&&X\ar[llld]^z\ar[lllu]_t\ar[ll]_\pi\\
\Pc
}
\]
Nach Bemerkung \ref{verzwmult} multiplizieren sich die Verzweigungsindizes beim Kompositum. $t$ ist unverzweigt in $P$, d.h. der Verzweigungsindex in $P$ ist $1$. Der Verzweigungsindex von $\pi$ in $P$ ist damit ebenfalls $1$. $z$ ist total verzweigt in $P$, d.h. der Verzweigungsindex in $P$ stimmt mit dem Grad von $z$ �berein. Da sich die Grade von endlichen Morphismen zwischen vollst�ndigen Kurven ebenfalls multiplizieren, ist auch $\pi$ in $P$ total verzweigt. Damit hat $\pi$ Grad $1$ und es gilt $[K(X):K(Y)]=1$.

Der lokale Parameter $s$ aus $\O_{X,P}$ wurde so gew�hlt, da� er invariant unter $U(X,t,P)$ ist. $\sigma\in U(X,t,P)$ induziert einen lokalen Ringisomorphismus $\O_{X,P}\longrightarrow\O_{X,P}$, es gilt $v_P(\textfrak z)=v_P(\sigma(\textfrak z))$. In Punkten $\not=P$ ist die Bewertung gr��er gleich $0$, also hat $\sigma(\textfrak z)$ einen Pol minimaler Ordnung in $P$ als einzigen Pol. Seien $a_i$ die Koeffizienten von $\sigma(\textfrak z)$, dann gilt $a_{-n}=1$ und $a_0=0$. Wir hatten vorher gesehen, da� $\textfrak z$ mit diesen Eigenschaften eindeutig bestimmt ist, also gilt $\sigma(\textfrak z)=\textfrak z$. Wir wissen nun, da� $\textfrak z$ invariant unter $U(X,t,P)$ ist. $\textfrak t$ ist per Konstruktion ebenfalls invariant unter $U(X,t,P)$\footnote{siehe Diagramm aus Bemerkung \ref{op_uxtp}}, d.h. $\textfrak t,\textfrak z\in K(X)^{U(X,t,P)}$.

Setze $k:=M(X,t,P)$. Es gilt $C(\textfrak t)^{U(X,t,P)}=k(\textfrak t)$. $\textfrak z$ ist algebraisch �ber $k(\textfrak t)$, denn $\textfrak z$ hat ein normiertes Minimalpolynom $m_{\textfrak z}=\sum a_iX^i\in C(\textfrak t)[X]$. F�r $\sigma\in U(X,t,P)$ gilt $\sigma(m_{\textfrak z})(\textfrak z)=\sum\sigma(a_i)\textfrak z^i=\sigma(\sum a_i\textfrak z^i)=\sigma(m_{\textfrak z}(\textfrak z))=0$. Also ist $\sigma(m_{\textfrak z})$ auch ein normiertes Minimalpolynom von $\textfrak z$ und damit gilt wegen der Eindeutigkeit des Minimalpolynoms $m_{\textfrak z}=\sigma(m_{\textfrak z})$. Daraus folgt $m_{\textfrak z}\in k(\textfrak t)[X]$.

Wir betrachten den Integrit�tsring $C[\textfrak t,\textfrak z]\cong C[X_1,X_2]/(m_{\textfrak z})$. Dann ist $k[\textfrak t,\textfrak z]$ ebenfalls ein Integrit�tsring und es gilt $k[\textfrak t,\textfrak z]\otimes_k C\cong C[\textfrak t,\textfrak z]$. Damit ist die affine Kurve $\Spec C[\textfrak t,\textfrak z]$ �ber $k$ definiert, $K(X)$ ist birational �quivalent zu dieser, also ist $K(X)$ �ber $k$ definiert. $t$ wird von dem \khom\ $t^*$ induziert, dieser wird wiederum von dem \khom\ $k(T)\longrightarrow k(\textfrak t,\textfrak z)$ induziert. Nach Lemma \ref{def_endliche_erw} ist die \KE\ $k:M(X,t)$ endlich algebraisch.\sieg
Als n�chstes suchen wir eine Absch�tzung f�r den Grad der \KE\ $k:M(X,t)$. Dazu betrachten wir die Beweise von Lemma \ref{def_endliche_erw} und Theorem \ref{koeck22}.
\begin{bem}
\label{gradabsch2}
Sei $d$ der Grad von $t$, und sei $a$ die Anzahl der Elemente in $\Aut(t)$, dann gilt $[M(X,t,P):M(X,t)]\leq\frac da$.
\end{bem}
\bew
Die Faser $t^{-1}(Q)$ hat nach Satz \ref{alles_gleichoft} genau $d$ Elemente, da $Q$ im Beweis von Theorem \ref{koeck22} so gew�hlt wurde, da� $Q$ kein kritischer Wert von $t$ ist. Wir betrachten nun $t^{-1}(Q)/\Aut(t)$. Sei $P\in t^{-1}(Q)$, dann gilt nach Lemma \ref{algcovgalois} genau dann $f(P)=P$ f�r $f\in\Aut(t)$, wenn $f=\id$ gilt. Nach dem Burnside Lemma hat $t^{-1}(Q)/\Aut(t)$ dann genau $\frac da$ Elemente. $U(X,t)$ operiert auf $t^{-1}(Q)/\Aut(t)$, der Stabilisator von $\overline P$ ist $U(X,t,P)$. Nach Bemerkung \ref{stabilisatormaster} ist der Index von $U(X,t,P)$ in $U(X,t)$ h�chstens $\frac da$. Die Kurve $X$ und der Morphismus $t$ sind �ber $k=M(X,t,P)$ definiert, damit gilt $\Aut(C:k)\subset U(X,t)$. Nach Lemma \ref{abgkriteri} ist $U(X,t)$ damit abgeschlossen, und nach Lemma \ref{indendldegendl} gilt $[M(X,t,P):M(X,t)]\leq[U(X,t,P):U(X,t)]\leq\frac da$.\sieg
Nun k�nnen wir aus den Bemerkungen \ref{gradabsch1} und \ref{gradabsch2} eine weitere Gradabsch�tzung folgern:
\begin{fol}
Sei $X$ eine vollst�ndige Kurve �ber $\C$, $t:X\longrightarrow\P1$ ein endlicher Morphismus vom Grad $d$, dessen kritische Werte in $\dreier$ liegen. Sei $a$ die Anzahl der Elemente in $\Aut(t)$, dann sind $X$ und $t$ �ber einem Zahlk�rper $k$ definiert mit
\[
[k:\Q]\leq\frac da M_d
\]
\end{fol}

\begin{thebibliography}{}
\bibitem[K�]{}
\textsc{Bernhard K�ck}, ``Belyi's Theorem Revisited'', arXiv:math.AG/0108222, 2001.
\bibitem[K�2]{}
\textsc{Bernhard K�ck}, ``Belyi's Theorem Revisited'', Beitr�ge zur Algebra und Geometrie, Volume \textbf{45}, 2004.
\bibitem[Ha]{Hartshorne}
\textsc{R. Hartshorne}, ``Algebraic Geometry'', Graduate Texts in Mathematics \textbf{52}, Springer-Verlag, New York 1977.
\bibitem[Fo]{Forster}
\textsc{O. Forster}, ``Lectures on Riemann Surfaces'', Graduate Texts in Mathematics \textbf{81}, Springer-Verlag, New York 1981.
\bibitem[Bo]{bosch}
\textsc{S. Bosch}, ``Algebra'', 4. Auflage, Springer-Verlag, Berlin 2001.
\bibitem[Wi]{}
\textsc{D. J. Winter}, ``The Structure of Fields'', Graduate Texts in Mathematics \textbf{16}, Springer-Verlag, New York 1974.
\bibitem[Ma]{}
\textsc{H. Matsumura}, ``Commutative algebra'', Second Edition, Mathematics lecture note series \textbf{56}, The Benjamin/Cumming Publishing Co., 1980.
\bibitem[Hal]{Hall}
\textsc{M. Hall}, ``The theory of groups'', Second Edition, Macmillian Co., New York 1968.
\bibitem[Bb]{bour}
\textsc{N. Bourbaki}, ``Algebra I'', Chapters 1-3, Elements of Mathematics, Addison-Wesley Publishing Co., Great Britain 1973.
\bibitem[EH]{}
\textsc{D. Eisenbud, J. Harris}, ``The Geometry of Schemes'', Graduate Texts in Mathematics \textbf{197}, Springer-Verlag, New York 2000.
\bibitem[L�t]{}
\textsc{W. L�tkebohmert}, ``Codierungstheorie'', 1. Auflage, Vieweg, 2003.
\bibitem[Wa]{}
\textsc{F. Warner}, ``Foundations of Differentiable Manifolds and Lie Groups'', Graduate Texts in Mathematics \textbf{94}, Springer-Verlag, New York 1983.
\bibitem[SZ]{Zieschang}
\textsc{R. St�cker, H. Zieschang}, ``Algebraische Topologie'', Mathematische Leitf�den, B. G. Teubner, Stuttgart 1988.
\end{thebibliography}

\end{document}
